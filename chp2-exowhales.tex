\label{chap2}
\section{\MakeUppercase{Observations of Whales in Kepler Data Using Exoacoustic Imaging Techniques}}

\subsection{Observing an Exowhale}
\label{exowhaleobs}

We observed a certain star with an exoplanet through a telescope. In Figure~\ref{whalefig1} we plot a theoretical star-whale system. 

\begin{figure}  % No [] option should make the figure take up its own page
\centering
\epsfig{width=\linewidth,file=whalefig1.eps}
\caption[Whales are Everywhere (short caption example)]{An illustration of a system containing a star, shown on the left, and a whale, shown on the right, separated by a distance $a$, not to scale. The star and whale lie at distances of $r_{\star}$ and $r_{p}$, respectively, from the barycenter of the system, which is marked via a ``+'' symbol. Similarly, the star and whale lie at distances of $s_{\star}$ and $s_{p}$, respectively, from the photocenter of the system, which is marked via a ``$\times$'' symbol. All distances are sky-projected distances along the semi-major axis of the system, and thus are independent of the system's inclination. Note that although in this illustration the photocenter is to the left of the barycenter, it can lie anywhere between the star and whale.}
\label{whalefig1}
\end{figure}

\begin{figure}[h]  % Adding [h] will tell latex to place the figure with text if it can and looks good. If it can't place it with text (too big), then it goes to the end of the chapter and you will have to go back to [] or []h!]
\centering
\epsfig{width=\linewidth,file=whalefig1.eps}
\caption[Whales are Everywhere (short caption example)]{An illustration of a system containing a star, shown on the left, and a whale, shown on the right, separated by a distance $a$, not to scale. The star and whale lie at distances of $r_{\star}$ and $r_{p}$, respectively, from the barycenter of the system, which is marked via a ``+'' symbol. Similarly, the star and whale lie at distances of $s_{\star}$ and $s_{p}$, respectively, from the photocenter of the system, which is marked via a ``$\times$'' symbol. All distances are sky-projected distances along the semi-major axis of the system, and thus are independent of the system's inclination. Note that although in this illustration the photocenter is to the left of the barycenter, it can lie anywhere between the star and whale.}
\label{whalefig1a}
\end{figure}

\begin{figure}[h!]  % Using [h!] will force it to go right here in the text, and be mroe aggressive about fitting text on the same page.
\centering
\epsfig{width=\linewidth,file=whalefig1.eps}
\caption[Whales are Everywhere (short caption example)]{An illustration of a system containing a star, shown on the left, and a whale, shown on the right, separated by a distance $a$, not to scale. The star and whale lie at distances of $r_{\star}$ and $r_{p}$, respectively, from the barycenter of the system, which is marked via a ``+'' symbol. Similarly, the star and whale lie at distances of $s_{\star}$ and $s_{p}$, respectively, from the photocenter of the system, which is marked via a ``$\times$'' symbol. All distances are sky-projected distances along the semi-major axis of the system, and thus are independent of the system's inclination. Note that although in this illustration the photocenter is to the left of the barycenter, it can lie anywhere between the star and whale.}
\label{whalefig1b}
\end{figure}

\tabletypesize{\scriptsize}
\begin{deluxetable}{ccccccccc}
  \tablewidth{0pt}
  \tablecaption{Currently Known Exoplanets with the Most Negative $\alpha_{WHALE}$ Values}
  \tablecolumns{9}
  \tablehead{Name & $D$ & $M_{\star}$ & $R_{\star}$ & $T_{\star}$ & $M_{p}$ & $R_{p}$ & $P$ & $\alpha_{WHALE}$ \\ & (pc) & (M$_{\sun}$) & (R$_{\sun}$) & (K) & (M$_{\rm J}$) & (R$_{\rm J}$) & (Days) & ($\mu$as)}
  \startdata
  \cutinhead{K Band (2.19 $\mu$m)}
  WASP-12 b & 427 & 1.28 & 1.63 & 6300 & 1.35 & 1.79 & 1.091 & -0.05 \\ 
WASP-19 b & 250 & 0.93 & 0.99 & 5500 & 1.11 & 1.39 & 0.789 & -0.05 \\ 
WASP-33 b & 115 & 1.50 & 1.44 & 7430 & 2.05 & 1.50 & 1.220 & -0.04 \\ 
55 Cnc e & 12 & 0.96 & 0.96 & 5234 & 0.03 & 0.19 & 0.737 & -0.01 \\ 
CoRoT-1 b & 480 & 0.95 & 1.11 & 5950 & 1.03 & 1.49 & 1.509 & -0.01 \\

  \cutinhead{L Band (3.45 $\mu$m)}
  HD 209458 b & 49 & 1.13 & 1.16 & 6065 & 0.69 & 1.36 & 3.525 & -0.23 \\ 
WASP-33 b & 115 & 1.50 & 1.44 & 7430 & 2.05 & 1.50 & 1.220 & -0.20 \\ 
WASP-19 b & 250 & 0.93 & 0.99 & 5500 & 1.11 & 1.39 & 0.789 & -0.15 \\ 
WASP-17 b & 300 & 1.19 & 1.20 & 6550 & 0.49 & 1.51 & 3.735 & -0.11 \\ 
WASP-12 b & 427 & 1.28 & 1.63 & 6300 & 1.35 & 1.79 & 1.091 & -0.10 \\ 

  \cutinhead{M Band (4.75 $\mu$m)}
  HD 209458 b & 49 & 1.13 & 1.16 & 6065 & 0.69 & 1.36 & 3.525 & -0.66 \\ 
HD 189733 b & 19 & 0.81 & 0.76 & 5040 & 1.14 & 1.14 & 2.219 & -0.47 \\ 
WASP-33 b & 115 & 1.50 & 1.44 & 7430 & 2.05 & 1.50 & 1.220 & -0.29 \\ 
WASP-19 b & 250 & 0.93 & 0.99 & 5500 & 1.11 & 1.39 & 0.789 & -0.21 \\ 
WASP-17 b & 300 & 1.19 & 1.20 & 6550 & 0.49 & 1.51 & 3.735 & -0.19 \\ 

  \cutinhead{N Band (10.0 $\mu$m)}
  HD 189733 b & 19 & 0.81 & 0.76 & 5040 & 1.14 & 1.14 & 2.219 & -3.04 \\ 
HD 209458 b & 49 & 1.13 & 1.16 & 6065 & 0.69 & 1.36 & 3.525 & -1.53 \\ 
Gliese 436 b & 10 & 0.45 & 0.46 & 3684 & 0.07 & 0.38 & 2.644 & -0.95 \\ 
WASP-34 b & 120 & 1.01 & 0.93 & 5700 & 0.58 & 1.22 & 4.318 & -0.64 \\ 
GJ 1214 b & 12 & 0.16 & 0.21 & 3026 & 0.02 & 0.24 & 1.580 & -0.59 

  \enddata
  \label{tab1}
\end{deluxetable}

\section{Further Whale Observations}

We get into detail about exowhales.

\subsection{Whales and You: What you Need to Know}

A subsection on whales and you.

\subsubsection{Whales: The Noisy Killer}

Everyone can hear you scream underwater.

\paragraph{Whale Colors}

A paragraph, (what you might want to label a subsubsubsection), and whale chromotography. We could go as deep in sections as a subparagraph, but, well, let's not.

Instead let's show a rotated deluxetable on whales with errorbars in Table~\ref{modelresultstab}.

\begin{deluxetable}{rcccccccccc}
  \rotate  % Comment out for portrait, leave uncommented for landscape
  \tablewidth{0pt}  % Gotta have this in every deluxetable
  \tabletypesize{\scriptsize}
%   \setlength{\tabcolsep}{5pt} % ONLY INCLUDE THIS LINE IF YOU NEED IT! Having it included squeezs column space slightly (smaller pt = tigheter tables) to make it fit in margins since its such a big table.
  \tablecaption{Modeling Results: Median Values and Associated 1$\sigma$ Uncertainties}
  \tablecolumns{11}
  \tablehead{KOI & $J$ & $r_{\rm sum}$ & $k$ & $i$ & $e$cos$w$ & $e$sin$w$ & $P$ & $T_{0}$ & $A_{L_{p}}$ & $\chi^{2}_{red}$\\ & & & & ($\degr$) & & & (Days) & (BJD-2450000) & & }
  \startdata
  \cutinhead{PDC Light Curve With Eccentricity Fixed to Zero}
     2.01 & $ $ 0.0108$^{_{+0.0013}}_{^{-0.0011}}$ & 0.254$^{_{+0.005}}_{^{-0.005}}$ & 0.0769$^{_{+0.0004}}_{^{-0.0004}}$ & 83.92$^{_{+0.7045}}_{^{-0.5848}}$ & $ $ 0.000$^{_{+0.000}}_{^{-0.000}}$ & $ $ 0.000$^{_{+0.000}}_{^{-0.000}}$ & 2.204732$^{_{+3.1\textrm{e-}06}}_{^{-3.0\textrm{e-}06}}$ & 4954.35796$^{_{+0.00013}}_{^{-0.00013}}$ & $ $ 0.421$^{_{+0.07}}_{^{-0.07}}$ & 26.5\\
   5.01 & $ $-0.0088$^{_{+0.0088}}_{^{-0.0097}}$ & 0.123$^{_{+0.010}}_{^{-0.013}}$ & 0.0356$^{_{+0.0009}}_{^{-0.0011}}$ & 83.60$^{_{+0.7978}}_{^{-0.5911}}$ & $ $ 0.000$^{_{+0.000}}_{^{-0.000}}$ & $ $ 0.000$^{_{+0.000}}_{^{-0.000}}$ & 4.780381$^{_{+6.7\textrm{e-}05}}_{^{-6.7\textrm{e-}05}}$ & 4965.97227$^{_{+0.00119}}_{^{-0.00121}}$ & $ $ 0.300$^{_{+0.72}}_{^{-0.35}}$ & 2.69\\
  10.01 & $ $ 0.0014$^{_{+0.0033}}_{^{-0.0013}}$ & 0.151$^{_{+0.004}}_{^{-0.004}}$ & 0.0938$^{_{+0.0006}}_{^{-0.0007}}$ & 84.68$^{_{+0.3495}}_{^{-0.2975}}$ & $ $ 0.000$^{_{+0.000}}_{^{-0.000}}$ & $ $ 0.000$^{_{+0.000}}_{^{-0.000}}$ & 3.522511$^{_{+1.5\textrm{e-}05}}_{^{-1.6\textrm{e-}05}}$ & 4954.11838$^{_{+0.00043}}_{^{-0.00041}}$ & $ $-5.752$^{_{+4.04}}_{^{-64.5}}$ & 4.31\\
  13.01 & $ $ 0.0248$^{_{+0.0013}}_{^{-0.0015}}$ & 0.294$^{_{+0.005}}_{^{-0.006}}$ & 0.0659$^{_{+0.0002}}_{^{-0.0003}}$ & 79.79$^{_{+0.5070}}_{^{-0.4079}}$ & $ $ 0.000$^{_{+0.000}}_{^{-0.000}}$ & $ $ 0.000$^{_{+0.000}}_{^{-0.000}}$ & 1.763589$^{_{+2.3\textrm{e-}06}}_{^{-2.3\textrm{e-}06}}$ & 4953.56510$^{_{+0.00013}}_{^{-0.00014}}$ & $ $ 0.339$^{_{+0.02}}_{^{-0.02}}$ & 15.9\\
  18.01 & $ $ 0.0000$^{_{+0.0001}}_{^{-0.0000}}$ & 0.174$^{_{+0.027}}_{^{-0.005}}$ & 0.0783$^{_{+0.0025}}_{^{-0.0007}}$ & 87.64$^{_{+1.7674}}_{^{-3.6176}}$ & $ $ 0.000$^{_{+0.000}}_{^{-0.000}}$ & $ $ 0.000$^{_{+0.000}}_{^{-0.000}}$ & 3.548460$^{_{+2.4\textrm{e-}05}}_{^{-2.5\textrm{e-}05}}$ & 4955.90081$^{_{+0.00068}}_{^{-0.00064}}$ & $ $-78.785$^{_{+238.}}_{^{-525.}}$ & 9.33\\
  64.01 & $ $ 0.0207$^{_{+0.0104}}_{^{-0.0100}}$ & 0.283$^{_{+0.018}}_{^{-0.019}}$ & 0.0425$^{_{+0.0018}}_{^{-0.0013}}$ & 75.01$^{_{+1.1206}}_{^{-1.0527}}$ & $ $ 0.000$^{_{+0.000}}_{^{-0.000}}$ & $ $ 0.000$^{_{+0.000}}_{^{-0.000}}$ & 1.951178$^{_{+2.8\textrm{e-}05}}_{^{-2.9\textrm{e-}05}}$ & 4990.53822$^{_{+0.00086}}_{^{-0.00086}}$ & $ $-0.312$^{_{+0.62}}_{^{-0.84}}$ & 10.4\\
  97.01 & $ $ 0.0066$^{_{+0.0017}}_{^{-0.0016}}$ & 0.156$^{_{+0.003}}_{^{-0.004}}$ & 0.0817$^{_{+0.0005}}_{^{-0.0005}}$ & 85.93$^{_{+0.4176}}_{^{-0.3637}}$ & $ $ 0.000$^{_{+0.000}}_{^{-0.000}}$ & $ $ 0.000$^{_{+0.000}}_{^{-0.000}}$ & 4.885495$^{_{+1.7\textrm{e-}05}}_{^{-1.7\textrm{e-}05}}$ & 4967.27590$^{_{+0.00030}}_{^{-0.00031}}$ & $ $ 0.115$^{_{+0.24}}_{^{-0.23}}$ & 2.24\\
 102.01 & $ $ 0.0140$^{_{+0.0089}}_{^{-0.0101}}$ & 0.199$^{_{+0.026}}_{^{-0.021}}$ & 0.0284$^{_{+0.0008}}_{^{-0.0007}}$ & 84.17$^{_{+2.8540}}_{^{-2.5667}}$ & $ $ 0.000$^{_{+0.000}}_{^{-0.000}}$ & $ $ 0.000$^{_{+0.000}}_{^{-0.000}}$ & 1.735114$^{_{+2.1\textrm{e-}05}}_{^{-2.1\textrm{e-}05}}$ & 4968.06072$^{_{+0.00101}}_{^{-0.00102}}$ & $ $-0.187$^{_{+0.34}}_{^{-0.56}}$ & 2.32\\
 144.01 & $ $ 0.0353$^{_{+0.0337}}_{^{-0.0292}}$ & 0.127$^{_{+0.026}}_{^{-0.012}}$ & 0.0352$^{_{+0.0021}}_{^{-0.0012}}$ & 86.86$^{_{+2.3598}}_{^{-2.5586}}$ & $ $ 0.000$^{_{+0.000}}_{^{-0.000}}$ & $ $ 0.000$^{_{+0.000}}_{^{-0.000}}$ & 4.176149$^{_{+1.5\textrm{e-}04}}_{^{-1.6\textrm{e-}04}}$ & 4966.09112$^{_{+0.00322}}_{^{-0.00310}}$ & $ $ 2.019$^{_{+5.88}}_{^{-1.25}}$ & 6.92\\
 186.01 & $ $-0.0014$^{_{+0.0026}}_{^{-0.0027}}$ & 0.126$^{_{+0.005}}_{^{-0.003}}$ & 0.1218$^{_{+0.0011}}_{^{-0.0008}}$ & 88.51$^{_{+1.3964}}_{^{-0.8215}}$ & $ $ 0.000$^{_{+0.000}}_{^{-0.000}}$ & $ $ 0.000$^{_{+0.000}}_{^{-0.000}}$ & 3.243268$^{_{+1.3\textrm{e-}05}}_{^{-1.4\textrm{e-}05}}$ & 4966.66796$^{_{+0.00037}}_{^{-0.00036}}$ & $ $ 0.051$^{_{+0.93}}_{^{-1.12}}$ & 3.20\\
 188.01 & $ $ 0.0033$^{_{+0.0024}}_{^{-0.0022}}$ & 0.092$^{_{+0.004}}_{^{-0.004}}$ & 0.1155$^{_{+0.0016}}_{^{-0.0018}}$ & 87.37$^{_{+0.4475}}_{^{-0.3527}}$ & $ $ 0.000$^{_{+0.000}}_{^{-0.000}}$ & $ $ 0.000$^{_{+0.000}}_{^{-0.000}}$ & 3.797023$^{_{+1.2\textrm{e-}05}}_{^{-1.2\textrm{e-}05}}$ & 4966.50793$^{_{+0.00029}}_{^{-0.00028}}$ & $ $-0.309$^{_{+0.43}}_{^{-0.68}}$ & 2.14\\
 195.01 & $ $ 0.0047$^{_{+0.0027}}_{^{-0.0030}}$ & 0.107$^{_{+0.004}}_{^{-0.004}}$ & 0.1165$^{_{+0.0011}}_{^{-0.0013}}$ & 86.42$^{_{+0.3423}}_{^{-0.2949}}$ & $ $ 0.000$^{_{+0.000}}_{^{-0.000}}$ & $ $ 0.000$^{_{+0.000}}_{^{-0.000}}$ & 3.217522$^{_{+1.2\textrm{e-}05}}_{^{-1.3\textrm{e-}05}}$ & 4966.63096$^{_{+0.00033}}_{^{-0.00032}}$ & $ $-0.132$^{_{+0.42}}_{^{-0.56}}$ & 2.38\\
 196.01 & $ $ 0.0060$^{_{+0.0019}}_{^{-0.0019}}$ & 0.221$^{_{+0.005}}_{^{-0.006}}$ & 0.1022$^{_{+0.0007}}_{^{-0.0008}}$ & 82.05$^{_{+0.4642}}_{^{-0.3994}}$ & $ $ 0.000$^{_{+0.000}}_{^{-0.000}}$ & $ $ 0.000$^{_{+0.000}}_{^{-0.000}}$ & 1.855561$^{_{+5.0\textrm{e-}06}}_{^{-5.4\textrm{e-}06}}$ & 4970.18013$^{_{+0.00024}}_{^{-0.00023}}$ & $ $-0.168$^{_{+0.19}}_{^{-0.25}}$ & 2.33\\
 199.01 & $ $ 0.0029$^{_{+0.0030}}_{^{-0.0032}}$ & 0.151$^{_{+0.006}}_{^{-0.007}}$ & 0.0963$^{_{+0.0009}}_{^{-0.0009}}$ & 86.81$^{_{+1.0287}}_{^{-0.7643}}$ & $ $ 0.000$^{_{+0.000}}_{^{-0.000}}$ & $ $ 0.000$^{_{+0.000}}_{^{-0.000}}$ & 3.268695$^{_{+1.7\textrm{e-}05}}_{^{-1.6\textrm{e-}05}}$ & 4970.48096$^{_{+0.00043}}_{^{-0.00044}}$ & $ $ 0.079$^{_{+0.41}}_{^{-0.44}}$ & 1.40\\
 201.01 & $ $-0.0059$^{_{+0.0051}}_{^{-0.0051}}$ & 0.093$^{_{+0.009}}_{^{-0.005}}$ & 0.0800$^{_{+0.0021}}_{^{-0.0012}}$ & 88.31$^{_{+1.5193}}_{^{-1.1075}}$ & $ $ 0.000$^{_{+0.000}}_{^{-0.000}}$ & $ $ 0.000$^{_{+0.000}}_{^{-0.000}}$ & 4.225373$^{_{+2.1\textrm{e-}05}}_{^{-2.2\textrm{e-}05}}$ & 4970.56037$^{_{+0.00039}}_{^{-0.00039}}$ & $ $ 1.145$^{_{+1.92}}_{^{-0.81}}$ & 3.60\\
 202.01 & $ $ 0.0053$^{_{+0.0023}}_{^{-0.0023}}$ & 0.225$^{_{+0.004}}_{^{-0.004}}$ & 0.1040$^{_{+0.0006}}_{^{-0.0006}}$ & 80.75$^{_{+0.2992}}_{^{-0.2744}}$ & $ $ 0.000$^{_{+0.000}}_{^{-0.000}}$ & $ $ 0.000$^{_{+0.000}}_{^{-0.000}}$ & 1.720867$^{_{+4.4\textrm{e-}06}}_{^{-4.7\textrm{e-}06}}$ & 4966.02007$^{_{+0.00023}}_{^{-0.00023}}$ & $ $-0.333$^{_{+0.22}}_{^{-0.38}}$ & 8.81\\
 204.01 & $ $-0.0019$^{_{+0.0032}}_{^{-0.0041}}$ & 0.151$^{_{+0.012}}_{^{-0.014}}$ & 0.0817$^{_{+0.0017}}_{^{-0.0022}}$ & 85.16$^{_{+1.4103}}_{^{-0.9589}}$ & $ $ 0.000$^{_{+0.000}}_{^{-0.000}}$ & $ $ 0.000$^{_{+0.000}}_{^{-0.000}}$ & 3.246708$^{_{+2.4\textrm{e-}05}}_{^{-2.6\textrm{e-}05}}$ & 4966.37897$^{_{+0.00067}}_{^{-0.00064}}$ & $ $ 1.579$^{_{+6.48}}_{^{-3.05}}$ & 2.08\\
 229.01 & $ $ 0.0000$^{_{+0.0004}}_{^{-0.0000}}$ & 0.400$^{_{+0.005}}_{^{-0.005}}$ & 0.2872$^{_{+0.0327}}_{^{-0.0289}}$ & 67.77$^{_{+0.3073}}_{^{-0.3087}}$ & $ $ 0.000$^{_{+0.000}}_{^{-0.000}}$ & $ $ 0.000$^{_{+0.000}}_{^{-0.000}}$ & 3.573190$^{_{+8.5\textrm{e-}05}}_{^{-8.7\textrm{e-}05}}$ & 4967.93368$^{_{+0.00196}}_{^{-0.00200}}$ & $ $-2.565$^{_{+3.04}}_{^{-27.4}}$ & 1.98\\
 356.01 & $ $ 0.0007$^{_{+0.0140}}_{^{-0.0115}}$ & 0.169$^{_{+0.055}}_{^{-0.026}}$ & 0.0334$^{_{+0.0026}}_{^{-0.0013}}$ & 84.64$^{_{+3.6556}}_{^{-4.5146}}$ & $ $ 0.000$^{_{+0.000}}_{^{-0.000}}$ & $ $ 0.000$^{_{+0.000}}_{^{-0.000}}$ & 1.827099$^{_{+3.1\textrm{e-}05}}_{^{-3.2\textrm{e-}05}}$ & 5003.52457$^{_{+0.00093}}_{^{-0.00093}}$ & $ $-0.007$^{_{+0.81}}_{^{-1.10}}$ & 1.52\\
 412.01 & $ $ 0.0063$^{_{+0.0116}}_{^{-0.0121}}$ & 0.100$^{_{+0.025}}_{^{-0.008}}$ & 0.0571$^{_{+0.0030}}_{^{-0.0012}}$ & 87.79$^{_{+1.7520}}_{^{-2.3886}}$ & $ $ 0.000$^{_{+0.000}}_{^{-0.000}}$ & $ $ 0.000$^{_{+0.000}}_{^{-0.000}}$ & 4.146994$^{_{+5.6\textrm{e-}05}}_{^{-5.7\textrm{e-}05}}$ & 5003.32536$^{_{+0.00073}}_{^{-0.00075}}$ & $ $-0.487$^{_{+1.49}}_{^{-2.00}}$ & 2.64\\
 421.01 & $ $-0.0059$^{_{+0.0048}}_{^{-0.0042}}$ & 0.076$^{_{+0.005}}_{^{-0.005}}$ & 0.1197$^{_{+0.0021}}_{^{-0.0024}}$ & 88.00$^{_{+0.5723}}_{^{-0.4216}}$ & $ $ 0.000$^{_{+0.000}}_{^{-0.000}}$ & $ $ 0.000$^{_{+0.000}}_{^{-0.000}}$ & 4.454225$^{_{+2.2\textrm{e-}05}}_{^{-2.3\textrm{e-}05}}$ & 5005.81890$^{_{+0.00028}}_{^{-0.00026}}$ & $ $ 0.137$^{_{+0.83}}_{^{-0.72}}$ & 4.01\\
 433.01 & $ $ 0.0001$^{_{+0.0162}}_{^{-0.0202}}$ & 0.352$^{_{+0.055}}_{^{-0.070}}$ & 0.2406$^{_{+0.1861}}_{^{-0.1474}}$ & 70.78$^{_{+4.5618}}_{^{-3.4365}}$ & $ $ 0.000$^{_{+0.000}}_{^{-0.000}}$ & $ $ 0.000$^{_{+0.000}}_{^{-0.000}}$ & 4.030457$^{_{+1.5\textrm{e-}04}}_{^{-1.4\textrm{e-}04}}$ & 5004.09086$^{_{+0.00179}}_{^{-0.00180}}$ & $ $ 0.033$^{_{+0.56}}_{^{-0.10}}$ & 7.51\\
 611.01 & $ $-0.0030$^{_{+0.0078}}_{^{-0.0075}}$ & 0.126$^{_{+0.009}}_{^{-0.006}}$ & 0.0760$^{_{+0.0077}}_{^{-0.0022}}$ & 83.75$^{_{+0.3507}}_{^{-0.5390}}$ & $ $ 0.000$^{_{+0.000}}_{^{-0.000}}$ & $ $ 0.000$^{_{+0.000}}_{^{-0.000}}$ & 3.251646$^{_{+2.8\textrm{e-}05}}_{^{-2.6\textrm{e-}05}}$ & 5004.05985$^{_{+0.00038}}_{^{-0.00040}}$ & $ $-0.039$^{_{+0.73}}_{^{-0.74}}$ & 2.12\\
 684.01 & $ $ 0.0802$^{_{+0.0491}}_{^{-0.0439}}$ & 0.066$^{_{+0.028}}_{^{-0.016}}$ & 0.0269$^{_{+0.0029}}_{^{-0.0019}}$ & 87.36$^{_{+1.6669}}_{^{-2.0032}}$ & $ $ 0.000$^{_{+0.000}}_{^{-0.000}}$ & $ $ 0.000$^{_{+0.000}}_{^{-0.000}}$ & 4.035328$^{_{+1.9\textrm{e-}04}}_{^{-1.8\textrm{e-}04}}$ & 5005.25386$^{_{+0.00223}}_{^{-0.00229}}$ & $ $-0.183$^{_{+0.17}}_{^{-0.32}}$ & 2.11\\
 760.01 & $ $ 0.0023$^{_{+0.0054}}_{^{-0.0052}}$ & 0.094$^{_{+0.003}}_{^{-0.003}}$ & 0.1055$^{_{+0.0013}}_{^{-0.0013}}$ & 85.85$^{_{+0.1909}}_{^{-0.1787}}$ & $ $ 0.000$^{_{+0.000}}_{^{-0.000}}$ & $ $ 0.000$^{_{+0.000}}_{^{-0.000}}$ & 4.959343$^{_{+4.4\textrm{e-}05}}_{^{-4.4\textrm{e-}05}}$ & 5005.25676$^{_{+0.00043}}_{^{-0.00045}}$ & $ $-0.300$^{_{+1.14}}_{^{-1.38}}$ & 1.83\\
 801.01 & $ $ 0.0061$^{_{+0.0067}}_{^{-0.0059}}$ & 0.221$^{_{+0.021}}_{^{-0.023}}$ & 0.0865$^{_{+0.0023}}_{^{-0.0025}}$ & 83.93$^{_{+3.0796}}_{^{-1.9812}}$ & $ $ 0.000$^{_{+0.000}}_{^{-0.000}}$ & $ $ 0.000$^{_{+0.000}}_{^{-0.000}}$ & 1.625498$^{_{+1.1\textrm{e-}05}}_{^{-1.1\textrm{e-}05}}$ & 5003.82699$^{_{+0.00035}}_{^{-0.00037}}$ & $ $-0.722$^{_{+0.87}}_{^{-2.05}}$ & 2.81\\
 809.01 & $ $-0.0023$^{_{+0.0041}}_{^{-0.0048}}$ & 0.163$^{_{+0.017}}_{^{-0.007}}$ & 0.1177$^{_{+0.0027}}_{^{-0.0011}}$ & 87.56$^{_{+2.0953}}_{^{-2.2216}}$ & $ $ 0.000$^{_{+0.000}}_{^{-0.000}}$ & $ $ 0.000$^{_{+0.000}}_{^{-0.000}}$ & 1.594742$^{_{+7.1\textrm{e-}06}}_{^{-7.1\textrm{e-}06}}$ & 5003.64754$^{_{+0.00024}}_{^{-0.00023}}$ & $ $ 0.779$^{_{+2.22}}_{^{-2.24}}$ & 3.47\\
 813.01 & $ $-0.0166$^{_{+0.0094}}_{^{-0.0091}}$ & 0.089$^{_{+0.017}}_{^{-0.012}}$ & 0.0897$^{_{+0.0042}}_{^{-0.0031}}$ & 87.62$^{_{+2.2020}}_{^{-1.5063}}$ & $ $ 0.000$^{_{+0.000}}_{^{-0.000}}$ & $ $ 0.000$^{_{+0.000}}_{^{-0.000}}$ & 3.895887$^{_{+5.2\textrm{e-}05}}_{^{-5.4\textrm{e-}05}}$ & 5003.52771$^{_{+0.00069}}_{^{-0.00066}}$ & $ $-0.070$^{_{+0.34}}_{^{-0.42}}$ & 1.72\\
 830.01 & $ $ 0.0024$^{_{+0.0031}}_{^{-0.0032}}$ & 0.099$^{_{+0.006}}_{^{-0.002}}$ & 0.1370$^{_{+0.0026}}_{^{-0.0010}}$ & 89.03$^{_{+0.8555}}_{^{-1.0199}}$ & $ $ 0.000$^{_{+0.000}}_{^{-0.000}}$ & $ $ 0.000$^{_{+0.000}}_{^{-0.000}}$ & 3.525638$^{_{+1.1\textrm{e-}05}}_{^{-1.1\textrm{e-}05}}$ & 5003.04715$^{_{+0.00015}}_{^{-0.00015}}$ & $ $-0.775$^{_{+1.88}}_{^{-2.58}}$ & 4.23\\
 838.01 & $ $ 0.0001$^{_{+0.0044}}_{^{-0.0028}}$ & 0.118$^{_{+0.009}}_{^{-0.010}}$ & 0.1047$^{_{+0.0235}}_{^{-0.0139}}$ & 84.18$^{_{+0.5842}}_{^{-0.5591}}$ & $ $ 0.000$^{_{+0.000}}_{^{-0.000}}$ & $ $ 0.000$^{_{+0.000}}_{^{-0.000}}$ & 4.859423$^{_{+8.1\textrm{e-}05}}_{^{-8.2\textrm{e-}05}}$ & 5006.01042$^{_{+0.00077}}_{^{-0.00082}}$ & $ $ 1.279$^{_{+100.}}_{^{-18.1}}$ & 7.50\\
 840.01 & $ $-0.0045$^{_{+0.0061}}_{^{-0.0061}}$ & 0.113$^{_{+0.008}}_{^{-0.008}}$ & 0.1048$^{_{+0.0024}}_{^{-0.0026}}$ & 85.44$^{_{+0.5689}}_{^{-0.4960}}$ & $ $ 0.000$^{_{+0.000}}_{^{-0.000}}$ & $ $ 0.000$^{_{+0.000}}_{^{-0.000}}$ & 3.040348$^{_{+2.2\textrm{e-}05}}_{^{-2.3\textrm{e-}05}}$ & 5002.94813$^{_{+0.00041}}_{^{-0.00042}}$ & $ $ 0.104$^{_{+1.14}}_{^{-1.13}}$ & 4.10\\
 843.01 & $ $-0.0007$^{_{+0.0052}}_{^{-0.0135}}$ & 0.132$^{_{+0.027}}_{^{-0.032}}$ & 0.0544$^{_{+0.0028}}_{^{-0.0034}}$ & 84.60$^{_{+2.7198}}_{^{-1.8763}}$ & $ $ 0.000$^{_{+0.000}}_{^{-0.000}}$ & $ $ 0.000$^{_{+0.000}}_{^{-0.000}}$ & 4.190630$^{_{+1.1\textrm{e-}04}}_{^{-1.1\textrm{e-}04}}$ & 5004.43891$^{_{+0.00134}}_{^{-0.00137}}$ & $ $ 0.935$^{_{+61.1}}_{^{-2.33}}$ & 1.51\\
 897.01 & $ $ 0.0032$^{_{+0.0036}}_{^{-0.0032}}$ & 0.159$^{_{+0.008}}_{^{-0.010}}$ & 0.1166$^{_{+0.0016}}_{^{-0.0019}}$ & 85.30$^{_{+0.9264}}_{^{-0.7186}}$ & $ $ 0.000$^{_{+0.000}}_{^{-0.000}}$ & $ $ 0.000$^{_{+0.000}}_{^{-0.000}}$ & 2.052344$^{_{+8.8\textrm{e-}06}}_{^{-8.7\textrm{e-}06}}$ & 5002.89010$^{_{+0.00022}}_{^{-0.00022}}$ & $ $-1.469$^{_{+1.28}}_{^{-4.03}}$ & 2.58\\
 908.01 & $ $ 0.0050$^{_{+0.0060}}_{^{-0.0052}}$ & 0.092$^{_{+0.012}}_{^{-0.004}}$ & 0.0839$^{_{+0.0026}}_{^{-0.0010}}$ & 88.63$^{_{+1.2397}}_{^{-1.4946}}$ & $ $ 0.000$^{_{+0.000}}_{^{-0.000}}$ & $ $ 0.000$^{_{+0.000}}_{^{-0.000}}$ & 4.708363$^{_{+4.8\textrm{e-}05}}_{^{-4.8\textrm{e-}05}}$ & 5004.44523$^{_{+0.00048}}_{^{-0.00046}}$ & $ $ 0.200$^{_{+0.78}}_{^{-0.61}}$ & 1.47\\
 913.01 & $ $ 0.0021$^{_{+0.0027}}_{^{-0.0027}}$ & 0.105$^{_{+0.004}}_{^{-0.002}}$ & 0.1240$^{_{+0.0012}}_{^{-0.0007}}$ & 89.05$^{_{+0.8784}}_{^{-0.8239}}$ & $ $ 0.000$^{_{+0.000}}_{^{-0.000}}$ & $ $ 0.000$^{_{+0.000}}_{^{-0.000}}$ & 4.082286$^{_{+2.0\textrm{e-}05}}_{^{-1.9\textrm{e-}05}}$ & 5002.63669$^{_{+0.00023}}_{^{-0.00023}}$ & $ $ 0.162$^{_{+1.88}}_{^{-1.85}}$ & 2.82\\
 931.01 & $ $ 0.0007$^{_{+0.0034}}_{^{-0.0010}}$ & 0.112$^{_{+0.009}}_{^{-0.003}}$ & 0.1201$^{_{+0.0024}}_{^{-0.0008}}$ & 88.67$^{_{+1.0903}}_{^{-1.4164}}$ & $ $ 0.000$^{_{+0.000}}_{^{-0.000}}$ & $ $ 0.000$^{_{+0.000}}_{^{-0.000}}$ & 3.855646$^{_{+2.3\textrm{e-}05}}_{^{-2.2\textrm{e-}05}}$ & 5003.67760$^{_{+0.00028}}_{^{-0.00030}}$ & $ $-2.995$^{_{+6.90}}_{^{-45.3}}$ & 2.63\\
 961.02 & $ $ 0.0017$^{_{+0.0198}}_{^{-0.0184}}$ & 0.292$^{_{+0.076}}_{^{-0.072}}$ & 0.0486$^{_{+0.0108}}_{^{-0.0050}}$ & 75.39$^{_{+4.6244}}_{^{-4.8234}}$ & $ $ 0.000$^{_{+0.000}}_{^{-0.000}}$ & $ $ 0.000$^{_{+0.000}}_{^{-0.000}}$ & 0.453296$^{_{+6.3\textrm{e-}06}}_{^{-6.3\textrm{e-}06}}$ & 4966.86709$^{_{+0.00116}}_{^{-0.00114}}$ & $ $ 0.048$^{_{+0.32}}_{^{-0.34}}$ & 1.95\\
 961.03 & $ $ 0.0159$^{_{+0.0472}}_{^{-0.0397}}$ & 0.060$^{_{+0.004}}_{^{-0.004}}$ & 0.4879$^{_{+1.0011}}_{^{-0.4024}}$ & 86.69$^{_{+0.2489}}_{^{-0.2260}}$ & $ $ 0.000$^{_{+0.000}}_{^{-0.000}}$ & $ $ 0.000$^{_{+0.000}}_{^{-0.000}}$ & 1.865033$^{_{+5.5\textrm{e-}05}}_{^{-5.8\textrm{e-}05}}$ & 4966.79647$^{_{+0.00259}}_{^{-0.00251}}$ & $ $ 0.000$^{_{+0.00}}_{^{-0.00}}$ & 2.30\\
1419.01 & $ $ 0.0131$^{_{+0.0145}}_{^{-0.0130}}$ & 0.322$^{_{+0.049}}_{^{-0.050}}$ & 0.0555$^{_{+0.0128}}_{^{-0.0034}}$ & 73.38$^{_{+3.0786}}_{^{-3.1111}}$ & $ $ 0.000$^{_{+0.000}}_{^{-0.000}}$ & $ $ 0.000$^{_{+0.000}}_{^{-0.000}}$ & 1.336074$^{_{+2.7\textrm{e-}05}}_{^{-2.8\textrm{e-}05}}$ & 5011.25875$^{_{+0.00087}}_{^{-0.00092}}$ & $ $-0.874$^{_{+0.85}}_{^{-2.18}}$ & 1.99\\
1459.01 & $ $ 0.0267$^{_{+0.0110}}_{^{-0.0127}}$ & 0.403$^{_{+0.044}}_{^{-0.061}}$ & 0.0930$^{_{+0.0367}}_{^{-0.0143}}$ & 69.77$^{_{+4.0056}}_{^{-3.0350}}$ & $ $ 0.000$^{_{+0.000}}_{^{-0.000}}$ & $ $ 0.000$^{_{+0.000}}_{^{-0.000}}$ & 0.692024$^{_{+7.7\textrm{e-}06}}_{^{-7.2\textrm{e-}06}}$ & 4966.11024$^{_{+0.00088}}_{^{-0.00092}}$ & $ $ 1.794$^{_{+1.71}}_{^{-0.97}}$ & 9.55\\

  \cutinhead{CLM Light Curve With Eccentricity Fixed to Zero}
     1.01 & $ $-0.0002$^{_{+0.0004}}_{^{-0.0004}}$ & 0.142$^{_{+0.000}}_{^{-0.000}}$ & 0.1250$^{_{+0.0002}}_{^{-0.0002}}$ & 83.90$^{_{+0.0307}}_{^{-0.0291}}$ & $ $ 0.000$^{_{+0.000}}_{^{-0.000}}$ & $ $ 0.000$^{_{+0.000}}_{^{-0.000}}$ & 2.470614$^{_{+1.3\textrm{e-}06}}_{^{-1.3\textrm{e-}06}}$ & 4955.76254$^{_{+0.00005}}_{^{-0.00005}}$ & $ $ 0.250$^{_{+0.80}}_{^{-0.80}}$ & 1.94\\
   2.01 & $ $ 0.0113$^{_{+0.0014}}_{^{-0.0013}}$ & 0.255$^{_{+0.005}}_{^{-0.005}}$ & 0.0766$^{_{+0.0004}}_{^{-0.0004}}$ & 83.74$^{_{+0.6427}}_{^{-0.5590}}$ & $ $ 0.000$^{_{+0.000}}_{^{-0.000}}$ & $ $ 0.000$^{_{+0.000}}_{^{-0.000}}$ & 2.204733$^{_{+3.1\textrm{e-}06}}_{^{-3.0\textrm{e-}06}}$ & 4954.35793$^{_{+0.00013}}_{^{-0.00014}}$ & $ $ 0.240$^{_{+0.06}}_{^{-0.04}}$ & 7.61\\
   5.01 & $ $-0.0021$^{_{+0.0079}}_{^{-0.0087}}$ & 0.114$^{_{+0.012}}_{^{-0.016}}$ & 0.0354$^{_{+0.0011}}_{^{-0.0013}}$ & 84.15$^{_{+0.9448}}_{^{-0.7226}}$ & $ $ 0.000$^{_{+0.000}}_{^{-0.000}}$ & $ $ 0.000$^{_{+0.000}}_{^{-0.000}}$ & 4.780406$^{_{+6.8\textrm{e-}05}}_{^{-7.0\textrm{e-}05}}$ & 4965.97184$^{_{+0.00120}}_{^{-0.00121}}$ & $ $ 0.238$^{_{+0.92}}_{^{-0.92}}$ & 2.08\\
  10.01 & $ $ 0.0019$^{_{+0.0030}}_{^{-0.0032}}$ & 0.159$^{_{+0.005}}_{^{-0.005}}$ & 0.0929$^{_{+0.0007}}_{^{-0.0007}}$ & 84.11$^{_{+0.3818}}_{^{-0.3455}}$ & $ $ 0.000$^{_{+0.000}}_{^{-0.000}}$ & $ $ 0.000$^{_{+0.000}}_{^{-0.000}}$ & 3.522496$^{_{+1.9\textrm{e-}05}}_{^{-1.9\textrm{e-}05}}$ & 4954.11867$^{_{+0.00052}}_{^{-0.00053}}$ & $ $-1.296$^{_{+2.76}}_{^{-4.42}}$ & 2.71\\
  13.01 & $ $ 0.0253$^{_{+0.0012}}_{^{-0.0016}}$ & 0.287$^{_{+0.005}}_{^{-0.005}}$ & 0.0656$^{_{+0.0002}}_{^{-0.0003}}$ & 80.36$^{_{+0.4700}}_{^{-0.4390}}$ & $ $ 0.000$^{_{+0.000}}_{^{-0.000}}$ & $ $ 0.000$^{_{+0.000}}_{^{-0.000}}$ & 1.763585$^{_{+1.6\textrm{e-}06}}_{^{-1.7\textrm{e-}06}}$ & 4953.56529$^{_{+0.00008}}_{^{-0.00009}}$ & $ $ 0.286$^{_{+0.02}}_{^{-0.01}}$ & 3.89\\
  17.01 & $ $-0.0015$^{_{+0.0015}}_{^{-0.0019}}$ & 0.165$^{_{+0.005}}_{^{-0.006}}$ & 0.0955$^{_{+0.0008}}_{^{-0.0009}}$ & 85.73$^{_{+0.6314}}_{^{-0.5396}}$ & $ $ 0.000$^{_{+0.000}}_{^{-0.000}}$ & $ $ 0.000$^{_{+0.000}}_{^{-0.000}}$ & 3.234699$^{_{+1.2\textrm{e-}05}}_{^{-1.2\textrm{e-}05}}$ & 4954.48575$^{_{+0.00035}}_{^{-0.00036}}$ & $ $ 1.249$^{_{+3.52}}_{^{-1.93}}$ & 2.65\\
  18.01 & $ $-0.0037$^{_{+0.0040}}_{^{-0.0044}}$ & 0.173$^{_{+0.016}}_{^{-0.004}}$ & 0.0771$^{_{+0.0015}}_{^{-0.0005}}$ & 87.86$^{_{+1.6741}}_{^{-2.5008}}$ & $ $ 0.000$^{_{+0.000}}_{^{-0.000}}$ & $ $ 0.000$^{_{+0.000}}_{^{-0.000}}$ & 3.548447$^{_{+2.6\textrm{e-}05}}_{^{-2.7\textrm{e-}05}}$ & 4955.90128$^{_{+0.00074}}_{^{-0.00072}}$ & $ $ 1.457$^{_{+2.51}}_{^{-3.06}}$ & 3.73\\
  20.01 & $ $ 0.0003$^{_{+0.0010}}_{^{-0.0005}}$ & 0.142$^{_{+0.002}}_{^{-0.002}}$ & 0.1174$^{_{+0.0005}}_{^{-0.0005}}$ & 88.17$^{_{+0.5236}}_{^{-0.4041}}$ & $ $ 0.000$^{_{+0.000}}_{^{-0.000}}$ & $ $ 0.000$^{_{+0.000}}_{^{-0.000}}$ & 4.437979$^{_{+1.1\textrm{e-}05}}_{^{-1.0\textrm{e-}05}}$ & 5004.00819$^{_{+0.00011}}_{^{-0.00011}}$ & $ $-2.544$^{_{+4.51}}_{^{-30.3}}$ & 1.57\\
  64.01 & $ $ 0.0312$^{_{+0.0117}}_{^{-0.0112}}$ & 0.304$^{_{+0.028}}_{^{-0.019}}$ & 0.0427$^{_{+0.0050}}_{^{-0.0014}}$ & 73.80$^{_{+1.1307}}_{^{-1.7207}}$ & $ $ 0.000$^{_{+0.000}}_{^{-0.000}}$ & $ $ 0.000$^{_{+0.000}}_{^{-0.000}}$ & 1.951144$^{_{+3.3\textrm{e-}05}}_{^{-3.5\textrm{e-}05}}$ & 4990.53863$^{_{+0.00112}}_{^{-0.00109}}$ & $ $-0.120$^{_{+0.39}}_{^{-0.51}}$ & 7.24\\
  97.01 & $ $ 0.0084$^{_{+0.0022}}_{^{-0.0022}}$ & 0.154$^{_{+0.004}}_{^{-0.004}}$ & 0.0808$^{_{+0.0005}}_{^{-0.0006}}$ & 86.17$^{_{+0.5356}}_{^{-0.4202}}$ & $ $ 0.000$^{_{+0.000}}_{^{-0.000}}$ & $ $ 0.000$^{_{+0.000}}_{^{-0.000}}$ & 4.885522$^{_{+2.0\textrm{e-}05}}_{^{-2.0\textrm{e-}05}}$ & 4967.27550$^{_{+0.00034}}_{^{-0.00034}}$ & $ $-0.174$^{_{+0.06}}_{^{-0.09}}$ & 2.06\\
 127.01 & $ $-0.0026$^{_{+0.0031}}_{^{-0.0037}}$ & 0.117$^{_{+0.008}}_{^{-0.008}}$ & 0.0982$^{_{+0.0018}}_{^{-0.0018}}$ & 87.45$^{_{+1.3596}}_{^{-0.9077}}$ & $ $ 0.000$^{_{+0.000}}_{^{-0.000}}$ & $ $ 0.000$^{_{+0.000}}_{^{-0.000}}$ & 3.578772$^{_{+2.1\textrm{e-}05}}_{^{-2.1\textrm{e-}05}}$ & 4967.02976$^{_{+0.00048}}_{^{-0.00048}}$ & $ $ 0.151$^{_{+0.87}}_{^{-0.93}}$ & 3.80\\
 128.01 & $ $-0.0007$^{_{+0.0006}}_{^{-0.0029}}$ & 0.116$^{_{+0.003}}_{^{-0.003}}$ & 0.0983$^{_{+0.0007}}_{^{-0.0008}}$ & 85.73$^{_{+0.2412}}_{^{-0.2261}}$ & $ $ 0.000$^{_{+0.000}}_{^{-0.000}}$ & $ $ 0.000$^{_{+0.000}}_{^{-0.000}}$ & 4.942764$^{_{+2.7\textrm{e-}05}}_{^{-2.6\textrm{e-}05}}$ & 4969.32912$^{_{+0.00043}}_{^{-0.00044}}$ & $ $ 3.129$^{_{+88.2}}_{^{-2.57}}$ & 5.15\\
 144.01 & $ $ 0.0306$^{_{+0.0323}}_{^{-0.0303}}$ & 0.127$^{_{+0.052}}_{^{-0.011}}$ & 0.0290$^{_{+0.0030}}_{^{-0.0011}}$ & 86.94$^{_{+2.4182}}_{^{-4.6598}}$ & $ $ 0.000$^{_{+0.000}}_{^{-0.000}}$ & $ $ 0.000$^{_{+0.000}}_{^{-0.000}}$ & 4.176232$^{_{+2.6\textrm{e-}04}}_{^{-2.3\textrm{e-}04}}$ & 4966.09100$^{_{+0.00508}}_{^{-0.00522}}$ & $ $-0.253$^{_{+0.42}}_{^{-0.75}}$ & 2.32\\
 183.01 & $ $ 0.0009$^{_{+0.0026}}_{^{-0.0024}}$ & 0.143$^{_{+0.005}}_{^{-0.006}}$ & 0.1160$^{_{+0.0010}}_{^{-0.0011}}$ & 86.89$^{_{+0.7468}}_{^{-0.5931}}$ & $ $ 0.000$^{_{+0.000}}_{^{-0.000}}$ & $ $ 0.000$^{_{+0.000}}_{^{-0.000}}$ & 2.684312$^{_{+9.8\textrm{e-}06}}_{^{-9.3\textrm{e-}06}}$ & 4966.35461$^{_{+0.00030}}_{^{-0.00031}}$ & $ $-0.485$^{_{+1.71}}_{^{-2.27}}$ & 3.36\\
 186.01 & $ $-0.0044$^{_{+0.0035}}_{^{-0.0035}}$ & 0.125$^{_{+0.016}}_{^{-0.003}}$ & 0.1161$^{_{+0.0032}}_{^{-0.0008}}$ & 88.53$^{_{+1.2341}}_{^{-2.1877}}$ & $ $ 0.000$^{_{+0.000}}_{^{-0.000}}$ & $ $ 0.000$^{_{+0.000}}_{^{-0.000}}$ & 3.243284$^{_{+2.0\textrm{e-}05}}_{^{-2.1\textrm{e-}05}}$ & 4966.66745$^{_{+0.00052}}_{^{-0.00055}}$ & $ $ 1.022$^{_{+1.42}}_{^{-0.49}}$ & 1.87\\
 188.01 & $ $-0.0034$^{_{+0.0031}}_{^{-0.0032}}$ & 0.096$^{_{+0.005}}_{^{-0.006}}$ & 0.1119$^{_{+0.0019}}_{^{-0.0023}}$ & 87.02$^{_{+0.5816}}_{^{-0.4387}}$ & $ $ 0.000$^{_{+0.000}}_{^{-0.000}}$ & $ $ 0.000$^{_{+0.000}}_{^{-0.000}}$ & 3.797013$^{_{+1.8\textrm{e-}05}}_{^{-1.9\textrm{e-}05}}$ & 4966.50811$^{_{+0.00042}}_{^{-0.00043}}$ & $ $ 0.682$^{_{+1.17}}_{^{-0.34}}$ & 1.69\\
 195.01 & $ $ 0.0000$^{_{+0.0016}}_{^{-0.0000}}$ & 0.120$^{_{+0.004}}_{^{-0.004}}$ & 0.1163$^{_{+0.0011}}_{^{-0.0012}}$ & 85.39$^{_{+0.2986}}_{^{-0.2548}}$ & $ $ 0.000$^{_{+0.000}}_{^{-0.000}}$ & $ $ 0.000$^{_{+0.000}}_{^{-0.000}}$ & 3.217557$^{_{+1.7\textrm{e-}05}}_{^{-1.7\textrm{e-}05}}$ & 4966.63025$^{_{+0.00043}}_{^{-0.00045}}$ & $ $-2.605$^{_{+5.29}}_{^{-183.}}$ & 1.36\\
 196.01 & $ $ 0.0053$^{_{+0.0027}}_{^{-0.0026}}$ & 0.222$^{_{+0.007}}_{^{-0.008}}$ & 0.0987$^{_{+0.0009}}_{^{-0.0011}}$ & 81.90$^{_{+0.6662}}_{^{-0.5586}}$ & $ $ 0.000$^{_{+0.000}}_{^{-0.000}}$ & $ $ 0.000$^{_{+0.000}}_{^{-0.000}}$ & 1.855556$^{_{+6.7\textrm{e-}06}}_{^{-6.5\textrm{e-}06}}$ & 4970.18029$^{_{+0.00029}}_{^{-0.00029}}$ & $ $-0.070$^{_{+0.13}}_{^{-0.20}}$ & 1.21\\
 201.01 & $ $-0.0015$^{_{+0.0030}}_{^{-0.0042}}$ & 0.094$^{_{+0.010}}_{^{-0.006}}$ & 0.0795$^{_{+0.0022}}_{^{-0.0013}}$ & 88.18$^{_{+1.6531}}_{^{-1.1858}}$ & $ $ 0.000$^{_{+0.000}}_{^{-0.000}}$ & $ $ 0.000$^{_{+0.000}}_{^{-0.000}}$ & 4.225404$^{_{+2.8\textrm{e-}05}}_{^{-2.7\textrm{e-}05}}$ & 4970.55976$^{_{+0.00050}}_{^{-0.00053}}$ & $ $ 1.572$^{_{+10.9}}_{^{-3.07}}$ & 1.91\\
 202.01 & $ $ 0.0040$^{_{+0.0026}}_{^{-0.0031}}$ & 0.231$^{_{+0.005}}_{^{-0.006}}$ & 0.0981$^{_{+0.0007}}_{^{-0.0007}}$ & 80.22$^{_{+0.3884}}_{^{-0.3450}}$ & $ $ 0.000$^{_{+0.000}}_{^{-0.000}}$ & $ $ 0.000$^{_{+0.000}}_{^{-0.000}}$ & 1.720865$^{_{+6.1\textrm{e-}06}}_{^{-6.3\textrm{e-}06}}$ & 4966.02018$^{_{+0.00032}}_{^{-0.00031}}$ & $ $ 0.041$^{_{+0.30}}_{^{-0.41}}$ & 1.47\\
 203.01 & $ $-0.0007$^{_{+0.0029}}_{^{-0.0029}}$ & 0.211$^{_{+0.008}}_{^{-0.009}}$ & 0.1286$^{_{+0.0014}}_{^{-0.0014}}$ & 86.30$^{_{+1.6250}}_{^{-1.0981}}$ & $ $ 0.000$^{_{+0.000}}_{^{-0.000}}$ & $ $ 0.000$^{_{+0.000}}_{^{-0.000}}$ & 1.485701$^{_{+4.4\textrm{e-}06}}_{^{-4.2\textrm{e-}06}}$ & 4965.79327$^{_{+0.00024}}_{^{-0.00025}}$ & $ $ 0.648$^{_{+3.78}}_{^{-2.04}}$ & 19.4\\
 204.01 & $ $ 0.0052$^{_{+0.0088}}_{^{-0.0085}}$ & 0.181$^{_{+0.019}}_{^{-0.027}}$ & 0.0737$^{_{+0.0022}}_{^{-0.0032}}$ & 82.70$^{_{+2.0921}}_{^{-1.2986}}$ & $ $ 0.000$^{_{+0.000}}_{^{-0.000}}$ & $ $ 0.000$^{_{+0.000}}_{^{-0.000}}$ & 3.246635$^{_{+5.5\textrm{e-}05}}_{^{-5.8\textrm{e-}05}}$ & 4966.38059$^{_{+0.00156}}_{^{-0.00149}}$ & $ $-0.759$^{_{+2.15}}_{^{-2.36}}$ & 3.28\\
 214.01 & $ $-0.0062$^{_{+0.0113}}_{^{-0.0121}}$ & 0.107$^{_{+0.024}}_{^{-0.016}}$ & 0.0873$^{_{+0.0570}}_{^{-0.0059}}$ & 84.86$^{_{+0.9593}}_{^{-1.5310}}$ & $ $ 0.000$^{_{+0.000}}_{^{-0.000}}$ & $ $ 0.000$^{_{+0.000}}_{^{-0.000}}$ & 3.311885$^{_{+5.1\textrm{e-}05}}_{^{-5.2\textrm{e-}05}}$ & 4964.74112$^{_{+0.00140}}_{^{-0.00138}}$ & $ $ 0.297$^{_{+1.53}}_{^{-0.94}}$ & 9.26\\
 217.01 & $ $-0.0035$^{_{+0.0048}}_{^{-0.0051}}$ & 0.308$^{_{+0.041}}_{^{-0.028}}$ & 0.9595$^{_{+0.5014}}_{^{-0.2870}}$ & 73.38$^{_{+1.7996}}_{^{-2.7018}}$ & $ $ 0.000$^{_{+0.000}}_{^{-0.000}}$ & $ $ 0.000$^{_{+0.000}}_{^{-0.000}}$ & 3.905132$^{_{+4.7\textrm{e-}05}}_{^{-5.0\textrm{e-}05}}$ & 4966.41317$^{_{+0.00103}}_{^{-0.00107}}$ & $ $ 0.001$^{_{+0.02}}_{^{-0.02}}$ & 14.1\\
 229.01 & $ $-0.0001$^{_{+0.0033}}_{^{-0.0124}}$ & 0.118$^{_{+0.030}}_{^{-0.010}}$ & 0.0481$^{_{+0.0028}}_{^{-0.0011}}$ & 87.30$^{_{+2.0647}}_{^{-2.8908}}$ & $ $ 0.000$^{_{+0.000}}_{^{-0.000}}$ & $ $ 0.000$^{_{+0.000}}_{^{-0.000}}$ & 3.573283$^{_{+9.2\textrm{e-}05}}_{^{-9.5\textrm{e-}05}}$ & 4967.93051$^{_{+0.00220}}_{^{-0.00214}}$ & $ $ 0.473$^{_{+16.3}}_{^{-5.24}}$ & 1.55\\
 254.01 & $ $-0.0043$^{_{+0.0070}}_{^{-0.0074}}$ & 0.117$^{_{+0.008}}_{^{-0.010}}$ & 0.1704$^{_{+0.0047}}_{^{-0.0054}}$ & 86.47$^{_{+0.8810}}_{^{-0.6047}}$ & $ $ 0.000$^{_{+0.000}}_{^{-0.000}}$ & $ $ 0.000$^{_{+0.000}}_{^{-0.000}}$ & 2.455267$^{_{+2.9\textrm{e-}05}}_{^{-2.8\textrm{e-}05}}$ & 5003.82069$^{_{+0.00067}}_{^{-0.00068}}$ & $ $ 0.567$^{_{+1.48}}_{^{-1.55}}$ & 6.44\\
 356.01 & $ $ 0.0492$^{_{+0.0240}}_{^{-0.0219}}$ & 0.144$^{_{+0.036}}_{^{-0.006}}$ & 0.0306$^{_{+0.0017}}_{^{-0.0006}}$ & 88.06$^{_{+1.4945}}_{^{-4.8438}}$ & $ $ 0.000$^{_{+0.000}}_{^{-0.000}}$ & $ $ 0.000$^{_{+0.000}}_{^{-0.000}}$ & 1.826966$^{_{+5.1\textrm{e-}05}}_{^{-5.0\textrm{e-}05}}$ & 5003.52754$^{_{+0.00143}}_{^{-0.00145}}$ & $ $-0.024$^{_{+0.14}}_{^{-0.18}}$ & 2.29\\
 412.01 & $ $ 0.0002$^{_{+0.0060}}_{^{-0.0001}}$ & 0.098$^{_{+0.031}}_{^{-0.008}}$ & 0.0477$^{_{+0.0032}}_{^{-0.0012}}$ & 87.86$^{_{+1.6992}}_{^{-2.8928}}$ & $ $ 0.000$^{_{+0.000}}_{^{-0.000}}$ & $ $ 0.000$^{_{+0.000}}_{^{-0.000}}$ & 4.146970$^{_{+1.3\textrm{e-}04}}_{^{-1.3\textrm{e-}04}}$ & 5003.32630$^{_{+0.00177}}_{^{-0.00176}}$ & $ $-8.920$^{_{+8.77}}_{^{-339.}}$ & 4.62\\
 421.01 & $ $-0.0010$^{_{+0.0046}}_{^{-0.0048}}$ & 0.076$^{_{+0.006}}_{^{-0.008}}$ & 0.1157$^{_{+0.0027}}_{^{-0.0038}}$ & 87.96$^{_{+1.0035}}_{^{-0.5585}}$ & $ $ 0.000$^{_{+0.000}}_{^{-0.000}}$ & $ $ 0.000$^{_{+0.000}}_{^{-0.000}}$ & 4.454260$^{_{+3.5\textrm{e-}05}}_{^{-3.5\textrm{e-}05}}$ & 5005.81883$^{_{+0.00041}}_{^{-0.00042}}$ & $ $ 0.687$^{_{+3.43}}_{^{-1.77}}$ & 2.07\\
 433.01 & $ $ 0.0001$^{_{+0.0051}}_{^{-0.0001}}$ & 0.094$^{_{+0.027}}_{^{-0.005}}$ & 0.0471$^{_{+0.0032}}_{^{-0.0012}}$ & 88.25$^{_{+1.4186}}_{^{-2.7352}}$ & $ $ 0.000$^{_{+0.000}}_{^{-0.000}}$ & $ $ 0.000$^{_{+0.000}}_{^{-0.000}}$ & 4.030278$^{_{+1.6\textrm{e-}04}}_{^{-1.6\textrm{e-}04}}$ & 5004.09410$^{_{+0.00190}}_{^{-0.00191}}$ & $ $-5.396$^{_{+18.5}}_{^{-237.}}$ & 1.58\\
 611.01 & $ $ 0.0025$^{_{+0.0093}}_{^{-0.0091}}$ & 0.154$^{_{+0.024}}_{^{-0.029}}$ & 0.1562$^{_{+0.1307}}_{^{-0.0764}}$ & 81.93$^{_{+1.8034}}_{^{-1.4920}}$ & $ $ 0.000$^{_{+0.000}}_{^{-0.000}}$ & $ $ 0.000$^{_{+0.000}}_{^{-0.000}}$ & 3.251641$^{_{+3.9\textrm{e-}05}}_{^{-4.1\textrm{e-}05}}$ & 5004.06018$^{_{+0.00058}}_{^{-0.00056}}$ & $ $-0.010$^{_{+0.13}}_{^{-0.22}}$ & 2.82\\
 667.01 & $ $ 0.0002$^{_{+0.0247}}_{^{-0.0018}}$ & 0.086$^{_{+0.038}}_{^{-0.009}}$ & 0.0775$^{_{+0.0089}}_{^{-0.0030}}$ & 87.96$^{_{+1.7162}}_{^{-3.0764}}$ & $ $ 0.000$^{_{+0.000}}_{^{-0.000}}$ & $ $ 0.000$^{_{+0.000}}_{^{-0.000}}$ & 4.305106$^{_{+2.4\textrm{e-}04}}_{^{-2.3\textrm{e-}04}}$ & 5003.45531$^{_{+0.00280}}_{^{-0.00282}}$ & $ $-0.353$^{_{+6.28}}_{^{-66.9}}$ & 2.33\\
 684.01 & $ $ 0.1004$^{_{+0.0638}}_{^{-0.0614}}$ & 0.049$^{_{+0.010}}_{^{-0.005}}$ & 0.0250$^{_{+0.0014}}_{^{-0.0012}}$ & 88.99$^{_{+0.7713}}_{^{-1.0734}}$ & $ $ 0.000$^{_{+0.000}}_{^{-0.000}}$ & $ $ 0.000$^{_{+0.000}}_{^{-0.000}}$ & 4.035349$^{_{+2.4\textrm{e-}04}}_{^{-2.5\textrm{e-}04}}$ & 5005.25460$^{_{+0.00308}}_{^{-0.00304}}$ & $ $ 0.015$^{_{+0.04}}_{^{-0.05}}$ & 1.86\\
 760.01 & $ $ 0.0020$^{_{+0.0068}}_{^{-0.0069}}$ & 0.096$^{_{+0.004}}_{^{-0.004}}$ & 0.1044$^{_{+0.0017}}_{^{-0.0017}}$ & 85.70$^{_{+0.2468}}_{^{-0.2116}}$ & $ $ 0.000$^{_{+0.000}}_{^{-0.000}}$ & $ $ 0.000$^{_{+0.000}}_{^{-0.000}}$ & 4.959295$^{_{+5.6\textrm{e-}05}}_{^{-5.6\textrm{e-}05}}$ & 5005.25690$^{_{+0.00056}}_{^{-0.00056}}$ & $ $-0.112$^{_{+0.56}}_{^{-0.49}}$ & 1.41\\
 767.01 & $ $ 0.0052$^{_{+0.0038}}_{^{-0.0038}}$ & 0.141$^{_{+0.006}}_{^{-0.007}}$ & 0.1224$^{_{+0.0015}}_{^{-0.0018}}$ & 85.95$^{_{+0.6974}}_{^{-0.5371}}$ & $ $ 0.000$^{_{+0.000}}_{^{-0.000}}$ & $ $ 0.000$^{_{+0.000}}_{^{-0.000}}$ & 2.816536$^{_{+1.5\textrm{e-}05}}_{^{-1.6\textrm{e-}05}}$ & 5003.96629$^{_{+0.00028}}_{^{-0.00028}}$ & $ $-0.163$^{_{+0.23}}_{^{-0.46}}$ & 1.81\\
 801.01 & $ $ 0.0130$^{_{+0.0074}}_{^{-0.0082}}$ & 0.207$^{_{+0.038}}_{^{-0.017}}$ & 0.0804$^{_{+0.0038}}_{^{-0.0017}}$ & 85.57$^{_{+4.3150}}_{^{-3.9176}}$ & $ $ 0.000$^{_{+0.000}}_{^{-0.000}}$ & $ $ 0.000$^{_{+0.000}}_{^{-0.000}}$ & 1.625528$^{_{+1.7\textrm{e-}05}}_{^{-1.6\textrm{e-}05}}$ & 5003.82578$^{_{+0.00053}}_{^{-0.00054}}$ & $ $-0.184$^{_{+0.24}}_{^{-0.47}}$ & 2.13\\
 809.01 & $ $ 0.0074$^{_{+0.0052}}_{^{-0.0051}}$ & 0.205$^{_{+0.014}}_{^{-0.017}}$ & 0.1153$^{_{+0.0021}}_{^{-0.0024}}$ & 83.03$^{_{+1.3933}}_{^{-1.0817}}$ & $ $ 0.000$^{_{+0.000}}_{^{-0.000}}$ & $ $ 0.000$^{_{+0.000}}_{^{-0.000}}$ & 1.594732$^{_{+1.1\textrm{e-}05}}_{^{-1.0\textrm{e-}05}}$ & 5003.64783$^{_{+0.00034}}_{^{-0.00034}}$ & $ $ 0.028$^{_{+0.19}}_{^{-0.26}}$ & 1.62\\
 813.01 & $ $-0.0113$^{_{+0.0170}}_{^{-0.0157}}$ & 0.081$^{_{+0.022}}_{^{-0.005}}$ & 0.0810$^{_{+0.0049}}_{^{-0.0016}}$ & 88.56$^{_{+1.1321}}_{^{-2.2494}}$ & $ $ 0.000$^{_{+0.000}}_{^{-0.000}}$ & $ $ 0.000$^{_{+0.000}}_{^{-0.000}}$ & 3.895873$^{_{+9.6\textrm{e-}05}}_{^{-9.4\textrm{e-}05}}$ & 5003.52774$^{_{+0.00122}}_{^{-0.00125}}$ & $ $ 0.116$^{_{+0.23}}_{^{-0.20}}$ & 1.84\\
 830.01 & $ $ 0.0012$^{_{+0.0026}}_{^{-0.0023}}$ & 0.099$^{_{+0.005}}_{^{-0.002}}$ & 0.1274$^{_{+0.0021}}_{^{-0.0011}}$ & 88.94$^{_{+0.9826}}_{^{-0.8831}}$ & $ $ 0.000$^{_{+0.000}}_{^{-0.000}}$ & $ $ 0.000$^{_{+0.000}}_{^{-0.000}}$ & 3.525644$^{_{+1.2\textrm{e-}05}}_{^{-1.3\textrm{e-}05}}$ & 5003.04701$^{_{+0.00019}}_{^{-0.00017}}$ & $ $-1.527$^{_{+3.24}}_{^{-8.31}}$ & 1.37\\
 838.01 & $ $-0.0151$^{_{+0.0116}}_{^{-0.0116}}$ & 0.152$^{_{+0.015}}_{^{-0.030}}$ & 0.1845$^{_{+0.0840}}_{^{-0.1044}}$ & 81.97$^{_{+1.9229}}_{^{-0.9304}}$ & $ $ 0.000$^{_{+0.000}}_{^{-0.000}}$ & $ $ 0.000$^{_{+0.000}}_{^{-0.000}}$ & 4.859266$^{_{+1.2\textrm{e-}04}}_{^{-1.2\textrm{e-}04}}$ & 5006.01125$^{_{+0.00113}}_{^{-0.00114}}$ & $ $-0.043$^{_{+0.04}}_{^{-0.22}}$ & 1.78\\
 840.01 & $ $-0.0033$^{_{+0.0074}}_{^{-0.0070}}$ & 0.112$^{_{+0.011}}_{^{-0.016}}$ & 0.0952$^{_{+0.0031}}_{^{-0.0043}}$ & 85.44$^{_{+1.1176}}_{^{-0.7342}}$ & $ $ 0.000$^{_{+0.000}}_{^{-0.000}}$ & $ $ 0.000$^{_{+0.000}}_{^{-0.000}}$ & 3.040330$^{_{+3.1\textrm{e-}05}}_{^{-2.9\textrm{e-}05}}$ & 5002.94851$^{_{+0.00051}}_{^{-0.00050}}$ & $ $ 0.179$^{_{+0.99}}_{^{-0.69}}$ & 2.33\\
 843.01 & $ $ 0.0269$^{_{+0.0223}}_{^{-0.0197}}$ & 0.140$^{_{+0.045}}_{^{-0.046}}$ & 0.0507$^{_{+0.0039}}_{^{-0.0044}}$ & 84.08$^{_{+4.3353}}_{^{-2.9613}}$ & $ $ 0.000$^{_{+0.000}}_{^{-0.000}}$ & $ $ 0.000$^{_{+0.000}}_{^{-0.000}}$ & 4.190464$^{_{+1.8\textrm{e-}04}}_{^{-1.8\textrm{e-}04}}$ & 5004.44160$^{_{+0.00207}}_{^{-0.00221}}$ & $ $-0.188$^{_{+0.18}}_{^{-0.37}}$ & 1.74\\
 897.01 & $ $ 0.0065$^{_{+0.0054}}_{^{-0.0055}}$ & 0.172$^{_{+0.011}}_{^{-0.013}}$ & 0.1163$^{_{+0.0021}}_{^{-0.0024}}$ & 84.24$^{_{+1.0754}}_{^{-0.8300}}$ & $ $ 0.000$^{_{+0.000}}_{^{-0.000}}$ & $ $ 0.000$^{_{+0.000}}_{^{-0.000}}$ & 2.052357$^{_{+1.5\textrm{e-}05}}_{^{-1.4\textrm{e-}05}}$ & 5002.88992$^{_{+0.00036}}_{^{-0.00039}}$ & $ $-0.020$^{_{+0.34}}_{^{-0.45}}$ & 2.27\\
 908.01 & $ $ 0.0176$^{_{+0.0103}}_{^{-0.0095}}$ & 0.093$^{_{+0.020}}_{^{-0.005}}$ & 0.0794$^{_{+0.0041}}_{^{-0.0013}}$ & 88.35$^{_{+1.3065}}_{^{-2.1367}}$ & $ $ 0.000$^{_{+0.000}}_{^{-0.000}}$ & $ $ 0.000$^{_{+0.000}}_{^{-0.000}}$ & 4.708360$^{_{+8.6\textrm{e-}05}}_{^{-8.2\textrm{e-}05}}$ & 5004.44498$^{_{+0.00085}}_{^{-0.00087}}$ & $ $-0.167$^{_{+0.11}}_{^{-0.24}}$ & 1.78\\
 913.01 & $ $ 0.0016$^{_{+0.0038}}_{^{-0.0041}}$ & 0.105$^{_{+0.007}}_{^{-0.003}}$ & 0.1098$^{_{+0.0017}}_{^{-0.0008}}$ & 88.83$^{_{+1.0892}}_{^{-1.1304}}$ & $ $ 0.000$^{_{+0.000}}_{^{-0.000}}$ & $ $ 0.000$^{_{+0.000}}_{^{-0.000}}$ & 4.082304$^{_{+3.0\textrm{e-}05}}_{^{-3.1\textrm{e-}05}}$ & 5002.63661$^{_{+0.00039}}_{^{-0.00038}}$ & $ $-0.702$^{_{+2.07}}_{^{-2.44}}$ & 1.82\\
1176.01 & $ $-0.0016$^{_{+0.0039}}_{^{-0.0038}}$ & 0.138$^{_{+0.011}}_{^{-0.014}}$ & 0.1268$^{_{+0.0034}}_{^{-0.0046}}$ & 86.29$^{_{+1.6953}}_{^{-0.9589}}$ & $ $ 0.000$^{_{+0.000}}_{^{-0.000}}$ & $ $ 0.000$^{_{+0.000}}_{^{-0.000}}$ & 1.973750$^{_{+9.4\textrm{e-}06}}_{^{-9.6\textrm{e-}06}}$ & 5011.68870$^{_{+0.00021}}_{^{-0.00021}}$ & $ $ 0.338$^{_{+1.19}}_{^{-0.97}}$ & 1.87\\
1419.01 & $ $ 0.0146$^{_{+0.0371}}_{^{-0.0304}}$ & 0.221$^{_{+0.103}}_{^{-0.068}}$ & 0.0475$^{_{+0.0060}}_{^{-0.0039}}$ & 80.02$^{_{+5.1744}}_{^{-6.7217}}$ & $ $ 0.000$^{_{+0.000}}_{^{-0.000}}$ & $ $ 0.000$^{_{+0.000}}_{^{-0.000}}$ & 1.336124$^{_{+5.1\textrm{e-}05}}_{^{-4.9\textrm{e-}05}}$ & 5011.25829$^{_{+0.00175}}_{^{-0.00165}}$ & $ $-0.439$^{_{+1.43}}_{^{-1.95}}$ & 1.91\\
1459.01 & $ $ 0.0327$^{_{+0.0157}}_{^{-0.0151}}$ & 0.243$^{_{+0.056}}_{^{-0.035}}$ & 0.0637$^{_{+0.0053}}_{^{-0.0033}}$ & 79.92$^{_{+2.7393}}_{^{-3.8427}}$ & $ $ 0.000$^{_{+0.000}}_{^{-0.000}}$ & $ $ 0.000$^{_{+0.000}}_{^{-0.000}}$ & 0.692019$^{_{+9.5\textrm{e-}06}}_{^{-9.5\textrm{e-}06}}$ & 4966.11081$^{_{+0.00113}}_{^{-0.00116}}$ & $ $ 2.865$^{_{+2.60}}_{^{-1.11}}$ & 3.50\\
1541.01 & $ $ 0.0494$^{_{+0.0084}}_{^{-0.0091}}$ & 0.192$^{_{+0.005}}_{^{-0.005}}$ & 0.1726$^{_{+0.0014}}_{^{-0.0017}}$ & 85.74$^{_{+0.5331}}_{^{-0.4268}}$ & $ $ 0.000$^{_{+0.000}}_{^{-0.000}}$ & $ $ 0.000$^{_{+0.000}}_{^{-0.000}}$ & 2.379290$^{_{+1.5\textrm{e-}05}}_{^{-1.5\textrm{e-}05}}$ & 4966.65042$^{_{+0.00050}}_{^{-0.00055}}$ & $ $ 0.143$^{_{+0.18}}_{^{-0.15}}$ & 29.8\\
1543.01 & $ $ 0.0266$^{_{+0.0225}}_{^{-0.0236}}$ & 0.158$^{_{+0.014}}_{^{-0.006}}$ & 0.1422$^{_{+0.0042}}_{^{-0.0024}}$ & 87.93$^{_{+1.8368}}_{^{-1.6934}}$ & $ $ 0.000$^{_{+0.000}}_{^{-0.000}}$ & $ $ 0.000$^{_{+0.000}}_{^{-0.000}}$ & 3.964266$^{_{+4.6\textrm{e-}05}}_{^{-4.4\textrm{e-}05}}$ & 4969.02968$^{_{+0.00091}}_{^{-0.00091}}$ & $ $-1.080$^{_{+1.05}}_{^{-3.26}}$ & 46.4\\

  \cutinhead{PDC Light Curve With Eccentricity Allowed To Vary}
     2.01 & $ $ 0.0106$^{_{+0.0013}}_{^{-0.0011}}$ & 0.270$^{_{+0.013}}_{^{-0.010}}$ & 0.0768$^{_{+0.0003}}_{^{-0.0004}}$ & 83.10$^{_{+0.8755}}_{^{-1.0221}}$ & $ $ 0.001$^{_{+0.002}}_{^{-0.002}}$ & $ $ 0.067$^{_{+0.043}}_{^{-0.034}}$ & 2.204732$^{_{+3.1\textrm{e-}06}}_{^{-3.0\textrm{e-}06}}$ & 4954.35797$^{_{+0.00013}}_{^{-0.00013}}$ & $ $ 0.442$^{_{+0.07}}_{^{-0.07}}$ & 26.5\\
   5.01 & $ $ 0.0077$^{_{+0.0211}}_{^{-0.0100}}$ & 0.126$^{_{+0.007}}_{^{-0.009}}$ & 0.0357$^{_{+0.0008}}_{^{-0.0009}}$ & 83.38$^{_{+0.7151}}_{^{-0.4133}}$ & $ $ 0.014$^{_{+0.028}}_{^{-0.025}}$ & $ $ 0.009$^{_{+0.040}}_{^{-0.058}}$ & 4.780380$^{_{+6.7\textrm{e-}05}}_{^{-6.6\textrm{e-}05}}$ & 4965.97242$^{_{+0.00127}}_{^{-0.00121}}$ & $ $-0.141$^{_{+0.31}}_{^{-1.11}}$ & 2.70\\
  10.01 & $ $ 0.0032$^{_{+0.0043}}_{^{-0.0027}}$ & 0.125$^{_{+0.012}}_{^{-0.006}}$ & 0.0938$^{_{+0.0006}}_{^{-0.0007}}$ & 86.30$^{_{+0.3346}}_{^{-0.6457}}$ & $ $-0.002$^{_{+0.024}}_{^{-0.023}}$ & $ $-0.188$^{_{+0.089}}_{^{-0.062}}$ & 3.522511$^{_{+1.5\textrm{e-}05}}_{^{-1.6\textrm{e-}05}}$ & 4954.11837$^{_{+0.00046}}_{^{-0.00046}}$ & $ $-2.726$^{_{+1.65}}_{^{-9.39}}$ & 4.31\\
  13.01 & $ $ 0.0250$^{_{+0.0013}}_{^{-0.0015}}$ & 0.295$^{_{+0.015}}_{^{-0.011}}$ & 0.0657$^{_{+0.0002}}_{^{-0.0003}}$ & 79.74$^{_{+0.8677}}_{^{-1.1544}}$ & $ $ 0.002$^{_{+0.001}}_{^{-0.001}}$ & $ $ 0.020$^{_{+0.049}}_{^{-0.032}}$ & 1.763589$^{_{+2.3\textrm{e-}06}}_{^{-2.3\textrm{e-}06}}$ & 4953.56511$^{_{+0.00013}}_{^{-0.00014}}$ & $ $ 0.341$^{_{+0.02}}_{^{-0.02}}$ & 15.9\\
  18.01 & $ $ 0.0023$^{_{+0.0092}}_{^{-0.0022}}$ & 0.171$^{_{+0.041}}_{^{-0.015}}$ & 0.0782$^{_{+0.0010}}_{^{-0.0006}}$ & 87.70$^{_{+1.7823}}_{^{-2.5452}}$ & $ $-0.018$^{_{+0.121}}_{^{-0.160}}$ & $ $-0.055$^{_{+0.197}}_{^{-0.111}}$ & 3.548460$^{_{+2.4\textrm{e-}05}}_{^{-2.6\textrm{e-}05}}$ & 4955.90075$^{_{+0.00088}}_{^{-0.00085}}$ & $ $-3.250$^{_{+2.41}}_{^{-30.4}}$ & 9.32\\
  64.01 & $ $ 0.0134$^{_{+0.0090}}_{^{-0.0092}}$ & 0.231$^{_{+0.015}}_{^{-0.012}}$ & 0.0427$^{_{+0.0027}}_{^{-0.0013}}$ & 80.10$^{_{+0.6878}}_{^{-1.0034}}$ & $ $-0.009$^{_{+0.012}}_{^{-0.013}}$ & $ $-0.226$^{_{+0.064}}_{^{-0.064}}$ & 1.951177$^{_{+2.8\textrm{e-}05}}_{^{-2.9\textrm{e-}05}}$ & 4990.53809$^{_{+0.00096}}_{^{-0.00098}}$ & $ $-0.423$^{_{+0.89}}_{^{-1.41}}$ & 10.4\\
  97.01 & $ $ 0.0070$^{_{+0.0017}}_{^{-0.0016}}$ & 0.155$^{_{+0.012}}_{^{-0.010}}$ & 0.0817$^{_{+0.0005}}_{^{-0.0005}}$ & 86.02$^{_{+0.6034}}_{^{-0.7233}}$ & $ $-0.000$^{_{+0.003}}_{^{-0.003}}$ & $ $-0.010$^{_{+0.077}}_{^{-0.071}}$ & 4.885495$^{_{+1.7\textrm{e-}05}}_{^{-1.7\textrm{e-}05}}$ & 4967.27590$^{_{+0.00030}}_{^{-0.00031}}$ & $ $ 0.115$^{_{+0.23}}_{^{-0.22}}$ & 2.24\\
 102.01 & $ $ 0.0255$^{_{+0.0095}}_{^{-0.0120}}$ & 0.179$^{_{+0.018}}_{^{-0.014}}$ & 0.0303$^{_{+0.0017}}_{^{-0.0014}}$ & 84.56$^{_{+0.7794}}_{^{-0.9164}}$ & $ $ 0.010$^{_{+0.009}}_{^{-0.009}}$ & $ $-0.402$^{_{+0.218}}_{^{-0.183}}$ & 1.735108$^{_{+2.1\textrm{e-}05}}_{^{-2.1\textrm{e-}05}}$ & 4968.06104$^{_{+0.00108}}_{^{-0.00105}}$ & $ $-0.090$^{_{+0.17}}_{^{-0.21}}$ & 2.32\\
 144.01 & $ $ 0.0712$^{_{+0.0482}}_{^{-0.0409}}$ & 0.263$^{_{+0.020}}_{^{-0.037}}$ & 0.0347$^{_{+0.0006}}_{^{-0.0005}}$ & 79.28$^{_{+3.8171}}_{^{-1.1387}}$ & $ $ 0.397$^{_{+0.086}}_{^{-0.058}}$ & $ $ 0.652$^{_{+0.084}}_{^{-0.173}}$ & 4.176174$^{_{+1.7\textrm{e-}04}}_{^{-1.7\textrm{e-}04}}$ & 4966.09365$^{_{+0.00376}}_{^{-0.00420}}$ & $ $ 1.016$^{_{+1.47}}_{^{-0.55}}$ & 6.89\\
 186.01 & $ $-0.0031$^{_{+0.0038}}_{^{-0.0033}}$ & 0.119$^{_{+0.014}}_{^{-0.013}}$ & 0.1224$^{_{+0.0011}}_{^{-0.0010}}$ & 88.35$^{_{+0.7668}}_{^{-0.7526}}$ & $ $-0.002$^{_{+0.028}}_{^{-0.029}}$ & $ $-0.075$^{_{+0.103}}_{^{-0.127}}$ & 3.243267$^{_{+1.3\textrm{e-}05}}_{^{-1.4\textrm{e-}05}}$ & 4966.66801$^{_{+0.00037}}_{^{-0.00036}}$ & $ $ 0.027$^{_{+0.46}}_{^{-0.71}}$ & 3.20\\
 188.01 & $ $ 0.0051$^{_{+0.0030}}_{^{-0.0030}}$ & 0.077$^{_{+0.009}}_{^{-0.009}}$ & 0.1153$^{_{+0.0016}}_{^{-0.0018}}$ & 88.16$^{_{+0.4271}}_{^{-0.4090}}$ & $ $-0.006$^{_{+0.008}}_{^{-0.008}}$ & $ $-0.174$^{_{+0.120}}_{^{-0.114}}$ & 3.797023$^{_{+1.3\textrm{e-}05}}_{^{-1.2\textrm{e-}05}}$ & 4966.50793$^{_{+0.00028}}_{^{-0.00028}}$ & $ $-0.212$^{_{+0.26}}_{^{-0.46}}$ & 2.14\\
 195.01 & $ $ 0.0071$^{_{+0.0034}}_{^{-0.0033}}$ & 0.094$^{_{+0.011}}_{^{-0.011}}$ & 0.1160$^{_{+0.0012}}_{^{-0.0013}}$ & 87.22$^{_{+0.5812}}_{^{-0.6459}}$ & $ $ 0.006$^{_{+0.006}}_{^{-0.007}}$ & $ $-0.113$^{_{+0.114}}_{^{-0.121}}$ & 3.217521$^{_{+1.2\textrm{e-}05}}_{^{-1.2\textrm{e-}05}}$ & 4966.63098$^{_{+0.00033}}_{^{-0.00033}}$ & $ $-0.083$^{_{+0.27}}_{^{-0.35}}$ & 2.38\\
 196.01 & $ $ 0.0073$^{_{+0.0021}}_{^{-0.0019}}$ & 0.202$^{_{+0.015}}_{^{-0.023}}$ & 0.1023$^{_{+0.0007}}_{^{-0.0008}}$ & 83.35$^{_{+1.3374}}_{^{-1.0172}}$ & $ $-0.008$^{_{+0.007}}_{^{-0.007}}$ & $ $-0.096$^{_{+0.084}}_{^{-0.116}}$ & 1.855561$^{_{+5.1\textrm{e-}06}}_{^{-5.3\textrm{e-}06}}$ & 4970.18009$^{_{+0.00024}}_{^{-0.00023}}$ & $ $-0.129$^{_{+0.16}}_{^{-0.18}}$ & 2.33\\
 199.01 & $ $ 0.0051$^{_{+0.0032}}_{^{-0.0051}}$ & 0.141$^{_{+0.017}}_{^{-0.017}}$ & 0.0963$^{_{+0.0009}}_{^{-0.0009}}$ & 87.34$^{_{+0.9351}}_{^{-0.9736}}$ & $ $ 0.026$^{_{+0.024}}_{^{-0.023}}$ & $ $-0.069$^{_{+0.116}}_{^{-0.135}}$ & 3.268695$^{_{+1.7\textrm{e-}05}}_{^{-1.6\textrm{e-}05}}$ & 4970.48100$^{_{+0.00043}}_{^{-0.00044}}$ & $ $ 0.068$^{_{+0.22}}_{^{-0.23}}$ & 1.40\\
 201.01 & $ $-0.0076$^{_{+0.0061}}_{^{-0.0053}}$ & 0.090$^{_{+0.015}}_{^{-0.008}}$ & 0.0796$^{_{+0.0018}}_{^{-0.0009}}$ & 88.72$^{_{+1.1788}}_{^{-1.2048}}$ & $ $-0.007$^{_{+0.009}}_{^{-0.009}}$ & $ $-0.010$^{_{+0.131}}_{^{-0.123}}$ & 4.225373$^{_{+2.1\textrm{e-}05}}_{^{-2.2\textrm{e-}05}}$ & 4970.56037$^{_{+0.00040}}_{^{-0.00039}}$ & $ $ 0.966$^{_{+1.47}}_{^{-0.62}}$ & 3.60\\
 202.01 & $ $ 0.0069$^{_{+0.0023}}_{^{-0.0023}}$ & 0.198$^{_{+0.009}}_{^{-0.012}}$ & 0.1038$^{_{+0.0006}}_{^{-0.0006}}$ & 82.79$^{_{+0.7640}}_{^{-0.6781}}$ & $ $-0.017$^{_{+0.006}}_{^{-0.007}}$ & $ $-0.125$^{_{+0.051}}_{^{-0.062}}$ & 1.720867$^{_{+4.4\textrm{e-}06}}_{^{-4.6\textrm{e-}06}}$ & 4966.01997$^{_{+0.00024}}_{^{-0.00024}}$ & $ $-0.238$^{_{+0.16}}_{^{-0.23}}$ & 8.81\\
 204.01 & $ $-0.0053$^{_{+0.0046}}_{^{-0.0056}}$ & 0.141$^{_{+0.013}}_{^{-0.015}}$ & 0.0817$^{_{+0.0017}}_{^{-0.0021}}$ & 85.87$^{_{+0.9701}}_{^{-0.7645}}$ & $ $-0.016$^{_{+0.025}}_{^{-0.027}}$ & $ $-0.073$^{_{+0.131}}_{^{-0.129}}$ & 3.246707$^{_{+2.4\textrm{e-}05}}_{^{-2.5\textrm{e-}05}}$ & 4966.37895$^{_{+0.00068}}_{^{-0.00069}}$ & $ $ 1.055$^{_{+2.87}}_{^{-0.64}}$ & 2.08\\
 229.01 & $ $ 0.0264$^{_{+0.0155}}_{^{-0.0168}}$ & 0.148$^{_{+0.022}}_{^{-0.018}}$ & 0.0500$^{_{+0.0014}}_{^{-0.0008}}$ & 85.95$^{_{+2.5816}}_{^{-2.2049}}$ & $ $-0.043$^{_{+0.014}}_{^{-0.013}}$ & $ $ 0.236$^{_{+0.134}}_{^{-0.153}}$ & 3.573177$^{_{+6.7\textrm{e-}05}}_{^{-7.0\textrm{e-}05}}$ & 4967.93350$^{_{+0.00156}}_{^{-0.00154}}$ & $ $-0.332$^{_{+0.37}}_{^{-0.80}}$ & 1.93\\
 356.01 & $ $ 0.0262$^{_{+0.0151}}_{^{-0.0201}}$ & 0.166$^{_{+0.042}}_{^{-0.014}}$ & 0.0347$^{_{+0.0035}}_{^{-0.0021}}$ & 84.64$^{_{+0.9385}}_{^{-2.1961}}$ & $ $-0.050$^{_{+0.021}}_{^{-0.026}}$ & $ $-0.215$^{_{+0.365}}_{^{-0.310}}$ & 1.827096$^{_{+3.3\textrm{e-}05}}_{^{-3.5\textrm{e-}05}}$ & 5003.52430$^{_{+0.00109}}_{^{-0.00116}}$ & $ $-0.029$^{_{+0.18}}_{^{-0.21}}$ & 1.52\\
 412.01 & $ $ 0.0152$^{_{+0.0151}}_{^{-0.0155}}$ & 0.082$^{_{+0.018}}_{^{-0.011}}$ & 0.0566$^{_{+0.0029}}_{^{-0.0008}}$ & 88.87$^{_{+1.0283}}_{^{-1.2331}}$ & $ $ 0.002$^{_{+0.021}}_{^{-0.019}}$ & $ $-0.202$^{_{+0.192}}_{^{-0.203}}$ & 4.146994$^{_{+5.6\textrm{e-}05}}_{^{-5.7\textrm{e-}05}}$ & 5003.32538$^{_{+0.00073}}_{^{-0.00076}}$ & $ $-0.373$^{_{+0.55}}_{^{-1.70}}$ & 2.64\\
 421.01 & $ $-0.0087$^{_{+0.0055}}_{^{-0.0044}}$ & 0.086$^{_{+0.009}}_{^{-0.012}}$ & 0.1188$^{_{+0.0025}}_{^{-0.0026}}$ & 87.62$^{_{+0.9498}}_{^{-0.5578}}$ & $ $ 0.005$^{_{+0.006}}_{^{-0.007}}$ & $ $ 0.127$^{_{+0.141}}_{^{-0.125}}$ & 4.454225$^{_{+2.2\textrm{e-}05}}_{^{-2.2\textrm{e-}05}}$ & 5005.81889$^{_{+0.00027}}_{^{-0.00027}}$ & $ $ 0.113$^{_{+0.54}}_{^{-0.45}}$ & 4.01\\
 433.01 & $ $ 0.0437$^{_{+0.0998}}_{^{-0.0339}}$ & 0.161$^{_{+0.076}}_{^{-0.028}}$ & 0.0529$^{_{+0.0015}}_{^{-0.0010}}$ & 84.46$^{_{+3.1427}}_{^{-2.2738}}$ & $ $ 0.269$^{_{+0.022}}_{^{-0.096}}$ & $ $ 0.412$^{_{+0.332}}_{^{-0.179}}$ & 4.030406$^{_{+1.1\textrm{e-}04}}_{^{-1.1\textrm{e-}04}}$ & 5004.09261$^{_{+0.00159}}_{^{-0.00148}}$ & $ $ 0.287$^{_{+0.76}}_{^{-0.43}}$ & 7.50\\
 611.01 & $ $ 0.0133$^{_{+0.0068}}_{^{-0.0067}}$ & 0.108$^{_{+0.007}}_{^{-0.005}}$ & 0.0758$^{_{+0.0077}}_{^{-0.0021}}$ & 85.39$^{_{+0.3427}}_{^{-0.4494}}$ & $ $-0.021$^{_{+0.005}}_{^{-0.005}}$ & $ $-0.157$^{_{+0.054}}_{^{-0.066}}$ & 3.251646$^{_{+2.8\textrm{e-}05}}_{^{-2.5\textrm{e-}05}}$ & 5004.05972$^{_{+0.00038}}_{^{-0.00041}}$ & $ $ 0.081$^{_{+0.25}}_{^{-0.26}}$ & 2.12\\
 684.01 & $ $ 0.0453$^{_{+0.0358}}_{^{-0.0314}}$ & 0.131$^{_{+0.015}}_{^{-0.013}}$ & 0.0351$^{_{+0.0207}}_{^{-0.0036}}$ & 84.59$^{_{+0.7313}}_{^{-1.0427}}$ & $ $ 0.002$^{_{+0.005}}_{^{-0.006}}$ & $ $-0.274$^{_{+0.154}}_{^{-0.164}}$ & 4.035281$^{_{+2.1\textrm{e-}04}}_{^{-2.1\textrm{e-}04}}$ & 5005.25327$^{_{+0.00252}}_{^{-0.00265}}$ & $ $-0.157$^{_{+0.16}}_{^{-0.24}}$ & 2.11\\
 760.01 & $ $ 0.0018$^{_{+0.0085}}_{^{-0.0065}}$ & 0.090$^{_{+0.006}}_{^{-0.006}}$ & 0.1055$^{_{+0.0013}}_{^{-0.0013}}$ & 86.16$^{_{+0.4756}}_{^{-0.4736}}$ & $ $ 0.085$^{_{+0.033}}_{^{-0.032}}$ & $ $-0.051$^{_{+0.074}}_{^{-0.067}}$ & 4.959343$^{_{+4.4\textrm{e-}05}}_{^{-4.4\textrm{e-}05}}$ & 5005.25710$^{_{+0.00048}}_{^{-0.00048}}$ & $ $-0.182$^{_{+0.69}}_{^{-2.58}}$ & 1.83\\
 801.01 & $ $ 0.0121$^{_{+0.0064}}_{^{-0.0082}}$ & 0.215$^{_{+0.030}}_{^{-0.021}}$ & 0.0837$^{_{+0.0024}}_{^{-0.0008}}$ & 87.23$^{_{+2.5490}}_{^{-3.4209}}$ & $ $-0.012$^{_{+0.024}}_{^{-0.026}}$ & $ $ 0.079$^{_{+0.123}}_{^{-0.139}}$ & 1.625498$^{_{+1.1\textrm{e-}05}}_{^{-1.1\textrm{e-}05}}$ & 5003.82697$^{_{+0.00035}}_{^{-0.00038}}$ & $ $-0.474$^{_{+0.37}}_{^{-0.82}}$ & 2.81\\
 809.01 & $ $-0.0142$^{_{+0.0054}}_{^{-0.0058}}$ & 0.151$^{_{+0.024}}_{^{-0.022}}$ & 0.1230$^{_{+0.0016}}_{^{-0.0019}}$ & 86.26$^{_{+0.9381}}_{^{-1.1708}}$ & $ $-0.113$^{_{+0.010}}_{^{-0.010}}$ & $ $-0.269$^{_{+0.160}}_{^{-0.140}}$ & 1.594742$^{_{+7.4\textrm{e-}06}}_{^{-7.5\textrm{e-}06}}$ & 5003.64725$^{_{+0.00026}}_{^{-0.00026}}$ & $ $ 0.387$^{_{+0.28}}_{^{-0.17}}$ & 3.46\\
 813.01 & $ $-0.0205$^{_{+0.0188}}_{^{-0.0106}}$ & 0.101$^{_{+0.026}}_{^{-0.018}}$ & 0.0876$^{_{+0.0031}}_{^{-0.0011}}$ & 88.09$^{_{+1.4899}}_{^{-2.8157}}$ & $ $-0.006$^{_{+0.016}}_{^{-0.045}}$ & $ $ 0.176$^{_{+0.171}}_{^{-0.221}}$ & 3.895881$^{_{+5.3\textrm{e-}05}}_{^{-5.5\textrm{e-}05}}$ & 5003.52771$^{_{+0.00078}}_{^{-0.00082}}$ & $ $-0.034$^{_{+0.28}}_{^{-0.36}}$ & 1.72\\
 830.01 & $ $ 0.0070$^{_{+0.0036}}_{^{-0.0059}}$ & 0.089$^{_{+0.023}}_{^{-0.012}}$ & 0.1370$^{_{+0.0016}}_{^{-0.0010}}$ & 89.20$^{_{+0.6777}}_{^{-0.9649}}$ & $ $-0.028$^{_{+0.007}}_{^{-0.013}}$ & $ $-0.126$^{_{+0.201}}_{^{-0.138}}$ & 3.525638$^{_{+1.0\textrm{e-}05}}_{^{-1.1\textrm{e-}05}}$ & 5003.04714$^{_{+0.00015}}_{^{-0.00017}}$ & $ $-0.524$^{_{+0.53}}_{^{-0.93}}$ & 4.22\\
 838.01 & $ $ 0.0001$^{_{+0.0007}}_{^{-0.0004}}$ & 0.054$^{_{+0.003}}_{^{-0.004}}$ & 0.1080$^{_{+0.0433}}_{^{-0.0191}}$ & 88.37$^{_{+0.1654}}_{^{-0.1793}}$ & $ $-0.079$^{_{+0.105}}_{^{-0.096}}$ & $ $-0.678$^{_{+0.087}}_{^{-0.069}}$ & 4.859425$^{_{+8.4\textrm{e-}05}}_{^{-8.7\textrm{e-}05}}$ & 5006.00988$^{_{+0.00111}}_{^{-0.00115}}$ & $ $ 9.589$^{_{+85.4}}_{^{-54.7}}$ & 7.51\\
 840.01 & $ $-0.0074$^{_{+0.0079}}_{^{-0.0078}}$ & 0.085$^{_{+0.009}}_{^{-0.009}}$ & 0.1049$^{_{+0.0023}}_{^{-0.0027}}$ & 87.34$^{_{+0.4691}}_{^{-0.5066}}$ & $ $-0.003$^{_{+0.018}}_{^{-0.018}}$ & $ $-0.285$^{_{+0.138}}_{^{-0.114}}$ & 3.040347$^{_{+2.2\textrm{e-}05}}_{^{-2.3\textrm{e-}05}}$ & 5002.94812$^{_{+0.00042}}_{^{-0.00043}}$ & $ $ 0.085$^{_{+0.82}}_{^{-0.77}}$ & 4.10\\
 843.01 & $ $-0.0005$^{_{+0.0022}}_{^{-0.0088}}$ & 0.112$^{_{+0.009}}_{^{-0.008}}$ & 0.0550$^{_{+0.0029}}_{^{-0.0036}}$ & 86.21$^{_{+0.6140}}_{^{-0.5054}}$ & $ $ 0.024$^{_{+0.160}}_{^{-0.164}}$ & $ $-0.240$^{_{+0.274}}_{^{-0.185}}$ & 4.190622$^{_{+1.2\textrm{e-}04}}_{^{-1.1\textrm{e-}04}}$ & 5004.43916$^{_{+0.00175}}_{^{-0.00170}}$ & $ $ 0.970$^{_{+17.3}}_{^{-9.81}}$ & 1.51\\
 897.01 & $ $ 0.0054$^{_{+0.0057}}_{^{-0.0048}}$ & 0.126$^{_{+0.017}}_{^{-0.017}}$ & 0.1154$^{_{+0.0018}}_{^{-0.0020}}$ & 87.18$^{_{+0.8279}}_{^{-0.7868}}$ & $ $ 0.011$^{_{+0.023}}_{^{-0.023}}$ & $ $-0.196$^{_{+0.150}}_{^{-0.135}}$ & 2.052344$^{_{+8.6\textrm{e-}06}}_{^{-8.7\textrm{e-}06}}$ & 5002.89012$^{_{+0.00024}}_{^{-0.00024}}$ & $ $-0.986$^{_{+0.73}}_{^{-3.16}}$ & 2.58\\
 908.01 & $ $ 0.0093$^{_{+0.0068}}_{^{-0.0092}}$ & 0.094$^{_{+0.025}}_{^{-0.014}}$ & 0.0844$^{_{+0.0021}}_{^{-0.0013}}$ & 88.28$^{_{+1.2959}}_{^{-1.7425}}$ & $ $-0.031$^{_{+0.020}}_{^{-0.051}}$ & $ $-0.017$^{_{+0.202}}_{^{-0.172}}$ & 4.708363$^{_{+4.9\textrm{e-}05}}_{^{-4.9\textrm{e-}05}}$ & 5004.44518$^{_{+0.00051}}_{^{-0.00056}}$ & $ $ 0.155$^{_{+0.52}}_{^{-0.31}}$ & 1.47\\
 913.01 & $ $ 0.0047$^{_{+0.0031}}_{^{-0.0040}}$ & 0.095$^{_{+0.012}}_{^{-0.009}}$ & 0.1241$^{_{+0.0013}}_{^{-0.0008}}$ & 89.18$^{_{+0.7440}}_{^{-0.7086}}$ & $ $-0.013$^{_{+0.014}}_{^{-0.013}}$ & $ $-0.103$^{_{+0.103}}_{^{-0.107}}$ & 4.082286$^{_{+2.0\textrm{e-}05}}_{^{-1.9\textrm{e-}05}}$ & 5002.63669$^{_{+0.00023}}_{^{-0.00022}}$ & $ $ 0.131$^{_{+0.96}}_{^{-0.92}}$ & 2.82\\
 931.01 & $ $ 0.0027$^{_{+0.0049}}_{^{-0.0025}}$ & 0.111$^{_{+0.020}}_{^{-0.009}}$ & 0.1200$^{_{+0.0013}}_{^{-0.0008}}$ & 88.76$^{_{+1.0292}}_{^{-1.3306}}$ & $ $-0.027$^{_{+0.035}}_{^{-0.076}}$ & $ $-0.025$^{_{+0.138}}_{^{-0.097}}$ & 3.855646$^{_{+2.3\textrm{e-}05}}_{^{-2.3\textrm{e-}05}}$ & 5003.67756$^{_{+0.00031}}_{^{-0.00035}}$ & $ $-1.988$^{_{+1.40}}_{^{-8.15}}$ & 2.63\\
 961.02 & $ $ 0.0026$^{_{+0.0404}}_{^{-0.0259}}$ & 0.298$^{_{+0.028}}_{^{-0.028}}$ & 0.0487$^{_{+0.0076}}_{^{-0.0046}}$ & 74.29$^{_{+1.4869}}_{^{-1.1549}}$ & $ $-0.034$^{_{+0.164}}_{^{-0.157}}$ & $ $ 0.036$^{_{+0.083}}_{^{-0.129}}$ & 0.453296$^{_{+6.3\textrm{e-}06}}_{^{-6.2\textrm{e-}06}}$ & 4966.86703$^{_{+0.00143}}_{^{-0.00144}}$ & $ $ 0.007$^{_{+0.23}}_{^{-0.22}}$ & 1.95\\
 961.03 & $ $ 0.0000$^{_{+0.0023}}_{^{-0.0015}}$ & 0.061$^{_{+0.005}}_{^{-0.005}}$ & 0.1002$^{_{+0.2972}}_{^{-0.0580}}$ & 86.58$^{_{+0.2826}}_{^{-1.6218}}$ & $ $-0.085$^{_{+0.583}}_{^{-0.384}}$ & $ $-0.017$^{_{+0.099}}_{^{-0.217}}$ & 1.865069$^{_{+6.2\textrm{e-}05}}_{^{-6.5\textrm{e-}05}}$ & 4966.79490$^{_{+0.00309}}_{^{-0.00294}}$ & $ $-0.010$^{_{+2.92}}_{^{-5.75}}$ & 2.30\\
1419.01 & $ $ 0.0249$^{_{+0.0172}}_{^{-0.0150}}$ & 0.279$^{_{+0.024}}_{^{-0.024}}$ & 0.0558$^{_{+0.0208}}_{^{-0.0032}}$ & 77.67$^{_{+1.2237}}_{^{-1.0487}}$ & $ $-0.059$^{_{+0.014}}_{^{-0.015}}$ & $ $-0.169$^{_{+0.089}}_{^{-0.125}}$ & 1.336077$^{_{+2.6\textrm{e-}05}}_{^{-2.7\textrm{e-}05}}$ & 5011.25756$^{_{+0.00102}}_{^{-0.00101}}$ & $ $-0.556$^{_{+0.24}}_{^{-0.50}}$ & 1.99\\
1459.01 & $ $ 0.0181$^{_{+0.0167}}_{^{-0.0103}}$ & 0.368$^{_{+0.021}}_{^{-0.026}}$ & 0.1166$^{_{+0.0424}}_{^{-0.0340}}$ & 74.20$^{_{+0.9911}}_{^{-1.3131}}$ & $ $ 0.082$^{_{+0.012}}_{^{-0.013}}$ & $ $-0.177$^{_{+0.071}}_{^{-0.082}}$ & 0.692025$^{_{+7.9\textrm{e-}06}}_{^{-7.4\textrm{e-}06}}$ & 4966.11150$^{_{+0.00100}}_{^{-0.00105}}$ & $ $ 1.872$^{_{+1.08}}_{^{-0.77}}$ & 9.54\\

  \cutinhead{CLM Light Curve With Eccentricity Allowed To Vary}
     1.01 & $ $-0.0004$^{_{+0.0007}}_{^{-0.0004}}$ & 0.131$^{_{+0.002}}_{^{-0.002}}$ & 0.1250$^{_{+0.0002}}_{^{-0.0002}}$ & 84.83$^{_{+0.1677}}_{^{-0.1817}}$ & $ $-0.010$^{_{+0.033}}_{^{-0.031}}$ & $ $-0.087$^{_{+0.019}}_{^{-0.018}}$ & 2.470614$^{_{+1.3\textrm{e-}06}}_{^{-1.3\textrm{e-}06}}$ & 4955.76250$^{_{+0.00015}}_{^{-0.00015}}$ & $ $ 0.210$^{_{+0.46}}_{^{-0.48}}$ & 1.94\\
   2.01 & $ $ 0.0112$^{_{+0.0014}}_{^{-0.0013}}$ & 0.264$^{_{+0.009}}_{^{-0.008}}$ & 0.0765$^{_{+0.0003}}_{^{-0.0003}}$ & 83.35$^{_{+0.7450}}_{^{-0.7427}}$ & $ $-0.001$^{_{+0.002}}_{^{-0.002}}$ & $ $ 0.038$^{_{+0.028}}_{^{-0.026}}$ & 2.204733$^{_{+3.1\textrm{e-}06}}_{^{-3.0\textrm{e-}06}}$ & 4954.35792$^{_{+0.00013}}_{^{-0.00014}}$ & $ $ 0.249$^{_{+0.06}}_{^{-0.04}}$ & 7.61\\
   5.01 & $ $-0.0000$^{_{+0.0080}}_{^{-0.0111}}$ & 0.136$^{_{+0.007}}_{^{-0.007}}$ & 0.0371$^{_{+0.0009}}_{^{-0.0009}}$ & 82.79$^{_{+0.5165}}_{^{-0.5651}}$ & $ $-0.015$^{_{+0.099}}_{^{-0.100}}$ & $ $-0.003$^{_{+0.045}}_{^{-0.046}}$ & 4.780376$^{_{+7.1\textrm{e-}05}}_{^{-7.1\textrm{e-}05}}$ & 4965.97212$^{_{+0.00183}}_{^{-0.00180}}$ & $ $ 0.021$^{_{+3.54}}_{^{-2.02}}$ & 2.08\\
  10.01 & $ $ 0.0043$^{_{+0.0042}}_{^{-0.0039}}$ & 0.126$^{_{+0.010}}_{^{-0.008}}$ & 0.0929$^{_{+0.0007}}_{^{-0.0007}}$ & 86.21$^{_{+0.4192}}_{^{-0.5845}}$ & $ $ 0.012$^{_{+0.021}}_{^{-0.019}}$ & $ $-0.231$^{_{+0.083}}_{^{-0.071}}$ & 3.522496$^{_{+1.9\textrm{e-}05}}_{^{-1.9\textrm{e-}05}}$ & 4954.11873$^{_{+0.00056}}_{^{-0.00054}}$ & $ $-0.844$^{_{+0.51}}_{^{-2.55}}$ & 2.71\\
  13.01 & $ $ 0.0255$^{_{+0.0012}}_{^{-0.0015}}$ & 0.309$^{_{+0.009}}_{^{-0.010}}$ & 0.0653$^{_{+0.0002}}_{^{-0.0003}}$ & 78.85$^{_{+0.7264}}_{^{-0.6362}}$ & $ $ 0.002$^{_{+0.001}}_{^{-0.001}}$ & $ $ 0.098$^{_{+0.053}}_{^{-0.042}}$ & 1.763585$^{_{+1.6\textrm{e-}06}}_{^{-1.6\textrm{e-}06}}$ & 4953.56529$^{_{+0.00008}}_{^{-0.00009}}$ & $ $ 0.292$^{_{+0.02}}_{^{-0.01}}$ & 3.88\\
  17.01 & $ $-0.0038$^{_{+0.0029}}_{^{-0.0025}}$ & 0.147$^{_{+0.015}}_{^{-0.016}}$ & 0.0941$^{_{+0.0009}}_{^{-0.0009}}$ & 87.29$^{_{+0.9074}}_{^{-0.8624}}$ & $ $ 0.020$^{_{+0.019}}_{^{-0.019}}$ & $ $-0.058$^{_{+0.102}}_{^{-0.117}}$ & 3.234699$^{_{+1.2\textrm{e-}05}}_{^{-1.1\textrm{e-}05}}$ & 4954.48579$^{_{+0.00034}}_{^{-0.00037}}$ & $ $ 0.772$^{_{+1.30}}_{^{-0.30}}$ & 2.65\\
  18.01 & $ $-0.0037$^{_{+0.0046}}_{^{-0.0060}}$ & 0.177$^{_{+0.042}}_{^{-0.023}}$ & 0.0771$^{_{+0.0011}}_{^{-0.0006}}$ & 87.71$^{_{+1.7534}}_{^{-2.8686}}$ & $ $ 0.014$^{_{+0.096}}_{^{-0.059}}$ & $ $-0.032$^{_{+0.209}}_{^{-0.153}}$ & 3.548447$^{_{+2.7\textrm{e-}05}}_{^{-2.8\textrm{e-}05}}$ & 4955.90133$^{_{+0.00094}}_{^{-0.00086}}$ & $ $ 1.118$^{_{+4.16}}_{^{-2.53}}$ & 3.73\\
  20.01 & $ $ 0.0007$^{_{+0.0016}}_{^{-0.0006}}$ & 0.132$^{_{+0.007}}_{^{-0.007}}$ & 0.1172$^{_{+0.0005}}_{^{-0.0005}}$ & 88.63$^{_{+0.5590}}_{^{-0.4472}}$ & $ $ 0.007$^{_{+0.039}}_{^{-0.037}}$ & $ $-0.068$^{_{+0.054}}_{^{-0.050}}$ & 4.437979$^{_{+1.1\textrm{e-}05}}_{^{-1.0\textrm{e-}05}}$ & 5004.00820$^{_{+0.00011}}_{^{-0.00012}}$ & $ $-1.722$^{_{+2.52}}_{^{-6.72}}$ & 1.57\\
  64.01 & $ $ 0.0196$^{_{+0.0104}}_{^{-0.0110}}$ & 0.268$^{_{+0.020}}_{^{-0.015}}$ & 0.0428$^{_{+0.0067}}_{^{-0.0016}}$ & 77.66$^{_{+0.8119}}_{^{-1.1367}}$ & $ $ 0.007$^{_{+0.009}}_{^{-0.010}}$ & $ $-0.151$^{_{+0.060}}_{^{-0.064}}$ & 1.951148$^{_{+3.3\textrm{e-}05}}_{^{-3.6\textrm{e-}05}}$ & 4990.53869$^{_{+0.00118}}_{^{-0.00112}}$ & $ $-0.188$^{_{+0.63}}_{^{-0.78}}$ & 7.25\\
  97.01 & $ $ 0.0088$^{_{+0.0021}}_{^{-0.0021}}$ & 0.167$^{_{+0.016}}_{^{-0.012}}$ & 0.0807$^{_{+0.0005}}_{^{-0.0006}}$ & 85.55$^{_{+0.8831}}_{^{-1.1023}}$ & $ $-0.005$^{_{+0.003}}_{^{-0.004}}$ & $ $ 0.084$^{_{+0.085}}_{^{-0.073}}$ & 4.885521$^{_{+2.0\textrm{e-}05}}_{^{-2.0\textrm{e-}05}}$ & 4967.27548$^{_{+0.00034}}_{^{-0.00034}}$ & $ $-0.156$^{_{+0.05}}_{^{-0.08}}$ & 2.05\\
 127.01 & $ $-0.0092$^{_{+0.0028}}_{^{-0.0033}}$ & 0.156$^{_{+0.015}}_{^{-0.012}}$ & 0.0966$^{_{+0.0008}}_{^{-0.0007}}$ & 87.18$^{_{+2.4895}}_{^{-1.7298}}$ & $ $ 0.041$^{_{+0.007}}_{^{-0.007}}$ & $ $ 0.341$^{_{+0.073}}_{^{-0.073}}$ & 3.578773$^{_{+2.1\textrm{e-}05}}_{^{-2.1\textrm{e-}05}}$ & 4967.02976$^{_{+0.00047}}_{^{-0.00047}}$ & $ $ 0.154$^{_{+0.20}}_{^{-0.28}}$ & 3.79\\
 128.01 & $ $-0.0001$^{_{+0.0000}}_{^{-0.0000}}$ & 0.086$^{_{+0.002}}_{^{-0.002}}$ & 0.0984$^{_{+0.0008}}_{^{-0.0008}}$ & 87.50$^{_{+0.0968}}_{^{-0.1243}}$ & $ $ 0.071$^{_{+0.152}}_{^{-0.187}}$ & $ $-0.311$^{_{+0.037}}_{^{-0.042}}$ & 4.942765$^{_{+2.7\textrm{e-}05}}_{^{-2.6\textrm{e-}05}}$ & 4969.32936$^{_{+0.00078}}_{^{-0.00082}}$ & $ $49.513$^{_{+86.1}}_{^{-49.6}}$ & 5.15\\
 144.01 & $ $ 0.0880$^{_{+0.0351}}_{^{-0.0314}}$ & 0.133$^{_{+0.022}}_{^{-0.016}}$ & 0.0305$^{_{+0.0033}}_{^{-0.0017}}$ & 85.83$^{_{+1.3423}}_{^{-1.2473}}$ & $ $ 0.035$^{_{+0.006}}_{^{-0.006}}$ & $ $-0.155$^{_{+0.209}}_{^{-0.265}}$ & 4.176281$^{_{+2.5\textrm{e-}04}}_{^{-2.4\textrm{e-}04}}$ & 4966.09046$^{_{+0.00506}}_{^{-0.00535}}$ & $ $-0.090$^{_{+0.11}}_{^{-0.15}}$ & 2.31\\
 183.01 & $ $ 0.0069$^{_{+0.0026}}_{^{-0.0025}}$ & 0.148$^{_{+0.020}}_{^{-0.019}}$ & 0.1159$^{_{+0.0011}}_{^{-0.0012}}$ & 86.78$^{_{+1.0868}}_{^{-1.2452}}$ & $ $-0.152$^{_{+0.009}}_{^{-0.008}}$ & $ $ 0.030$^{_{+0.124}}_{^{-0.138}}$ & 2.684313$^{_{+9.7\textrm{e-}06}}_{^{-9.2\textrm{e-}06}}$ & 4966.35439$^{_{+0.00031}}_{^{-0.00031}}$ & $ $-0.203$^{_{+0.10}}_{^{-0.18}}$ & 3.35\\
 186.01 & $ $-0.0066$^{_{+0.0054}}_{^{-0.0043}}$ & 0.122$^{_{+0.021}}_{^{-0.014}}$ & 0.1161$^{_{+0.0013}}_{^{-0.0007}}$ & 88.66$^{_{+1.1558}}_{^{-1.4415}}$ & $ $-0.005$^{_{+0.017}}_{^{-0.019}}$ & $ $-0.035$^{_{+0.134}}_{^{-0.123}}$ & 3.243285$^{_{+2.0\textrm{e-}05}}_{^{-2.0\textrm{e-}05}}$ & 4966.66741$^{_{+0.00052}}_{^{-0.00055}}$ & $ $ 0.780$^{_{+1.10}}_{^{-0.32}}$ & 1.86\\
 188.01 & $ $-0.0003$^{_{+0.0009}}_{^{-0.0030}}$ & 0.075$^{_{+0.005}}_{^{-0.004}}$ & 0.1080$^{_{+0.0024}}_{^{-0.0022}}$ & 88.52$^{_{+0.6189}}_{^{-0.3824}}$ & $ $ 0.103$^{_{+0.094}}_{^{-0.093}}$ & $ $-0.147$^{_{+0.092}}_{^{-0.098}}$ & 3.797011$^{_{+1.8\textrm{e-}05}}_{^{-1.8\textrm{e-}05}}$ & 4966.50823$^{_{+0.00042}}_{^{-0.00043}}$ & $ $ 0.957$^{_{+9.93}}_{^{-5.83}}$ & 1.69\\
 195.01 & $ $ 0.0081$^{_{+0.0047}}_{^{-0.0047}}$ & 0.081$^{_{+0.010}}_{^{-0.010}}$ & 0.1163$^{_{+0.0012}}_{^{-0.0012}}$ & 87.75$^{_{+0.4441}}_{^{-0.4550}}$ & $ $ 0.018$^{_{+0.007}}_{^{-0.007}}$ & $ $-0.379$^{_{+0.108}}_{^{-0.114}}$ & 3.217557$^{_{+1.7\textrm{e-}05}}_{^{-1.7\textrm{e-}05}}$ & 4966.63031$^{_{+0.00044}}_{^{-0.00045}}$ & $ $-0.242$^{_{+0.12}}_{^{-0.29}}$ & 1.36\\
 196.01 & $ $ 0.0066$^{_{+0.0032}}_{^{-0.0029}}$ & 0.180$^{_{+0.019}}_{^{-0.019}}$ & 0.0988$^{_{+0.0010}}_{^{-0.0011}}$ & 84.60$^{_{+0.9580}}_{^{-1.0762}}$ & $ $ 0.002$^{_{+0.011}}_{^{-0.011}}$ & $ $-0.217$^{_{+0.109}}_{^{-0.103}}$ & 1.855556$^{_{+6.6\textrm{e-}06}}_{^{-6.6\textrm{e-}06}}$ & 4970.18030$^{_{+0.00030}}_{^{-0.00030}}$ & $ $-0.055$^{_{+0.10}}_{^{-0.15}}$ & 1.21\\
 201.01 & $ $-0.0034$^{_{+0.0034}}_{^{-0.0059}}$ & 0.098$^{_{+0.013}}_{^{-0.008}}$ & 0.0789$^{_{+0.0022}}_{^{-0.0008}}$ & 88.49$^{_{+1.2420}}_{^{-1.4910}}$ & $ $ 0.031$^{_{+0.043}}_{^{-0.033}}$ & $ $ 0.056$^{_{+0.099}}_{^{-0.117}}$ & 4.225405$^{_{+2.8\textrm{e-}05}}_{^{-2.7\textrm{e-}05}}$ & 4970.55978$^{_{+0.00052}}_{^{-0.00055}}$ & $ $ 1.104$^{_{+5.04}}_{^{-1.54}}$ & 1.91\\
 202.01 & $ $ 0.0053$^{_{+0.0026}}_{^{-0.0028}}$ & 0.209$^{_{+0.006}}_{^{-0.010}}$ & 0.0981$^{_{+0.0007}}_{^{-0.0007}}$ & 82.02$^{_{+0.6930}}_{^{-0.4738}}$ & $ $-0.011$^{_{+0.015}}_{^{-0.015}}$ & $ $-0.105$^{_{+0.039}}_{^{-0.054}}$ & 1.720865$^{_{+6.1\textrm{e-}06}}_{^{-6.3\textrm{e-}06}}$ & 4966.02012$^{_{+0.00035}}_{^{-0.00035}}$ & $ $ 0.043$^{_{+0.21}}_{^{-0.26}}$ & 1.47\\
 203.01 & $ $ 0.0046$^{_{+0.0035}}_{^{-0.0046}}$ & 0.180$^{_{+0.034}}_{^{-0.022}}$ & 0.1287$^{_{+0.0013}}_{^{-0.0014}}$ & 87.23$^{_{+1.1967}}_{^{-1.2664}}$ & $ $ 0.093$^{_{+0.031}}_{^{-0.030}}$ & $ $-0.169$^{_{+0.168}}_{^{-0.138}}$ & 1.485701$^{_{+4.4\textrm{e-}06}}_{^{-4.2\textrm{e-}06}}$ & 4965.79338$^{_{+0.00025}}_{^{-0.00027}}$ & $ $-0.301$^{_{+0.67}}_{^{-1.14}}$ & 19.4\\
 204.01 & $ $ 0.0182$^{_{+0.0132}}_{^{-0.0157}}$ & 0.118$^{_{+0.018}}_{^{-0.018}}$ & 0.0693$^{_{+0.0036}}_{^{-0.0026}}$ & 87.19$^{_{+1.4075}}_{^{-1.0848}}$ & $ $-0.031$^{_{+0.016}}_{^{-0.018}}$ & $ $-0.224$^{_{+0.186}}_{^{-0.197}}$ & 3.246636$^{_{+5.4\textrm{e-}05}}_{^{-5.5\textrm{e-}05}}$ & 4966.38047$^{_{+0.00153}}_{^{-0.00150}}$ & $ $-0.428$^{_{+0.25}}_{^{-1.01}}$ & 3.28\\
 214.01 & $ $-0.0233$^{_{+0.0192}}_{^{-0.1543}}$ & 0.109$^{_{+0.008}}_{^{-0.010}}$ & 0.0884$^{_{+0.0325}}_{^{-0.0057}}$ & 84.68$^{_{+0.6625}}_{^{-0.5261}}$ & $ $-0.019$^{_{+0.008}}_{^{-0.009}}$ & $ $ 0.015$^{_{+0.063}}_{^{-0.113}}$ & 3.311884$^{_{+5.1\textrm{e-}05}}_{^{-5.3\textrm{e-}05}}$ & 4964.74106$^{_{+0.00139}}_{^{-0.00141}}$ & $ $ 0.183$^{_{+0.65}}_{^{-0.18}}$ & 9.26\\
 217.01 & $ $ 0.0190$^{_{+0.0073}}_{^{-0.0072}}$ & 0.102$^{_{+0.016}}_{^{-0.012}}$ & 0.1081$^{_{+0.0031}}_{^{-0.0011}}$ & 88.63$^{_{+1.2886}}_{^{-1.6338}}$ & $ $ 0.082$^{_{+0.006}}_{^{-0.007}}$ & $ $ 0.032$^{_{+0.121}}_{^{-0.140}}$ & 3.905094$^{_{+3.7\textrm{e-}05}}_{^{-3.7\textrm{e-}05}}$ & 4966.41399$^{_{+0.00079}}_{^{-0.00082}}$ & $ $ 0.114$^{_{+0.35}}_{^{-0.31}}$ & 13.8\\
 229.01 & $ $ 0.0030$^{_{+0.0230}}_{^{-0.0056}}$ & 0.133$^{_{+0.023}}_{^{-0.016}}$ & 0.0479$^{_{+0.0034}}_{^{-0.0009}}$ & 86.91$^{_{+2.3589}}_{^{-2.0024}}$ & $ $ 0.020$^{_{+0.092}}_{^{-0.087}}$ & $ $ 0.090$^{_{+0.191}}_{^{-0.292}}$ & 3.573280$^{_{+9.2\textrm{e-}05}}_{^{-9.5\textrm{e-}05}}$ & 4967.93062$^{_{+0.00228}}_{^{-0.00223}}$ & $ $-0.248$^{_{+1.34}}_{^{-3.54}}$ & 1.55\\
 254.01 & $ $ 0.0349$^{_{+0.0168}}_{^{-0.0123}}$ & 0.069$^{_{+0.017}}_{^{-0.013}}$ & 0.1713$^{_{+0.0045}}_{^{-0.0052}}$ & 88.62$^{_{+0.3976}}_{^{-0.5933}}$ & $ $ 0.052$^{_{+0.006}}_{^{-0.006}}$ & $ $-0.496$^{_{+0.194}}_{^{-0.135}}$ & 2.455264$^{_{+2.9\textrm{e-}05}}_{^{-2.8\textrm{e-}05}}$ & 5003.82083$^{_{+0.00065}}_{^{-0.00069}}$ & $ $-0.126$^{_{+0.06}}_{^{-0.10}}$ & 6.42\\
 356.01 & $ $ 0.0701$^{_{+0.0349}}_{^{-0.0312}}$ & 0.139$^{_{+0.041}}_{^{-0.026}}$ & 0.0310$^{_{+0.0030}}_{^{-0.0009}}$ & 86.93$^{_{+2.3117}}_{^{-2.7558}}$ & $ $-0.003$^{_{+0.011}}_{^{-0.010}}$ & $ $-0.174$^{_{+0.247}}_{^{-0.296}}$ & 1.826970$^{_{+5.3\textrm{e-}05}}_{^{-5.2\textrm{e-}05}}$ & 5003.52752$^{_{+0.00150}}_{^{-0.00159}}$ & $ $-0.005$^{_{+0.11}}_{^{-0.11}}$ & 2.29\\
 412.01 & $ $-0.0002$^{_{+0.0005}}_{^{-0.0012}}$ & 0.119$^{_{+0.038}}_{^{-0.014}}$ & 0.0477$^{_{+0.0036}}_{^{-0.0011}}$ & 86.47$^{_{+2.5213}}_{^{-2.7070}}$ & $ $ 0.013$^{_{+0.512}}_{^{-0.512}}$ & $ $ 0.026$^{_{+0.242}}_{^{-0.321}}$ & 4.146984$^{_{+1.4\textrm{e-}04}}_{^{-1.4\textrm{e-}04}}$ & 5003.32622$^{_{+0.00295}}_{^{-0.00288}}$ & $ $ 8.721$^{_{+51.4}}_{^{-33.8}}$ & 4.62\\
 421.01 & $ $ 0.0003$^{_{+0.0036}}_{^{-0.0009}}$ & 0.063$^{_{+0.006}}_{^{-0.004}}$ & 0.1134$^{_{+0.0031}}_{^{-0.0023}}$ & 88.83$^{_{+0.6649}}_{^{-0.4695}}$ & $ $ 0.011$^{_{+0.117}}_{^{-0.119}}$ & $ $-0.118$^{_{+0.098}}_{^{-0.125}}$ & 4.454248$^{_{+3.5\textrm{e-}05}}_{^{-3.4\textrm{e-}05}}$ & 5005.81896$^{_{+0.00042}}_{^{-0.00042}}$ & $ $-1.164$^{_{+7.39}}_{^{-11.3}}$ & 2.07\\
 433.01 & $ $-0.0002$^{_{+0.0068}}_{^{-0.0229}}$ & 0.086$^{_{+0.013}}_{^{-0.015}}$ & 0.0541$^{_{+0.0058}}_{^{-0.0056}}$ & 87.30$^{_{+0.6371}}_{^{-0.6889}}$ & $ $-0.231$^{_{+0.110}}_{^{-0.115}}$ & $ $-0.610$^{_{+0.332}}_{^{-0.276}}$ & 4.030290$^{_{+1.9\textrm{e-}04}}_{^{-1.8\textrm{e-}04}}$ & 5004.09156$^{_{+0.00284}}_{^{-0.01005}}$ & $ $ 0.135$^{_{+8.59}}_{^{-4.72}}$ & 1.58\\
 611.01 & $ $-0.0018$^{_{+0.0015}}_{^{-0.0055}}$ & 0.109$^{_{+0.005}}_{^{-0.005}}$ & 0.1796$^{_{+0.1310}}_{^{-0.0758}}$ & 85.78$^{_{+0.3957}}_{^{-0.3992}}$ & $ $ 0.059$^{_{+0.011}}_{^{-0.010}}$ & $ $-0.371$^{_{+0.136}}_{^{-0.107}}$ & 3.251642$^{_{+3.9\textrm{e-}05}}_{^{-4.0\textrm{e-}05}}$ & 5004.06072$^{_{+0.00063}}_{^{-0.00061}}$ & $ $ 0.124$^{_{+0.23}}_{^{-0.16}}$ & 2.82\\
 667.01 & $ $ 0.0003$^{_{+0.0146}}_{^{-0.0049}}$ & 0.099$^{_{+0.051}}_{^{-0.020}}$ & 0.0766$^{_{+0.0081}}_{^{-0.0026}}$ & 87.68$^{_{+1.7035}}_{^{-2.8017}}$ & $ $-0.256$^{_{+0.282}}_{^{-0.236}}$ & $ $ 0.029$^{_{+0.360}}_{^{-0.412}}$ & 4.305101$^{_{+2.6\textrm{e-}04}}_{^{-2.5\textrm{e-}04}}$ & 5003.45495$^{_{+0.00322}}_{^{-0.00353}}$ & $ $-0.137$^{_{+7.53}}_{^{-10.0}}$ & 2.33\\
 684.01 & $ $ 0.0388$^{_{+0.0504}}_{^{-0.0729}}$ & 0.080$^{_{+0.013}}_{^{-0.011}}$ & 0.0307$^{_{+0.0057}}_{^{-0.0042}}$ & 87.08$^{_{+0.5216}}_{^{-0.7949}}$ & $ $ 0.040$^{_{+0.029}}_{^{-0.026}}$ & $ $-0.378$^{_{+0.385}}_{^{-0.236}}$ & 4.035404$^{_{+2.6\textrm{e-}04}}_{^{-2.5\textrm{e-}04}}$ & 5005.25403$^{_{+0.00337}}_{^{-0.00319}}$ & $ $ 0.017$^{_{+0.06}}_{^{-0.06}}$ & 1.87\\
 760.01 & $ $ 0.0001$^{_{+0.0092}}_{^{-0.0079}}$ & 0.091$^{_{+0.005}}_{^{-0.005}}$ & 0.1048$^{_{+0.0018}}_{^{-0.0017}}$ & 86.18$^{_{+0.3778}}_{^{-0.4901}}$ & $ $ 0.002$^{_{+0.137}}_{^{-0.136}}$ & $ $-0.085$^{_{+0.074}}_{^{-0.062}}$ & 4.959296$^{_{+5.7\textrm{e-}05}}_{^{-5.6\textrm{e-}05}}$ & 5005.25691$^{_{+0.00084}}_{^{-0.00084}}$ & $ $-0.044$^{_{+1.51}}_{^{-3.02}}$ & 1.41\\
 767.01 & $ $ 0.0111$^{_{+0.0044}}_{^{-0.0043}}$ & 0.122$^{_{+0.019}}_{^{-0.017}}$ & 0.1224$^{_{+0.0016}}_{^{-0.0017}}$ & 86.95$^{_{+0.7904}}_{^{-0.9652}}$ & $ $ 0.012$^{_{+0.006}}_{^{-0.007}}$ & $ $-0.146$^{_{+0.145}}_{^{-0.141}}$ & 2.816536$^{_{+1.5\textrm{e-}05}}_{^{-1.6\textrm{e-}05}}$ & 5003.96631$^{_{+0.00028}}_{^{-0.00028}}$ & $ $-0.075$^{_{+0.10}}_{^{-0.14}}$ & 1.81\\
 801.01 & $ $ 0.0221$^{_{+0.0117}}_{^{-0.0103}}$ & 0.153$^{_{+0.026}}_{^{-0.022}}$ & 0.0798$^{_{+0.0039}}_{^{-0.0012}}$ & 88.08$^{_{+1.7431}}_{^{-2.0569}}$ & $ $-0.004$^{_{+0.012}}_{^{-0.012}}$ & $ $-0.289$^{_{+0.140}}_{^{-0.179}}$ & 1.625529$^{_{+1.7\textrm{e-}05}}_{^{-1.6\textrm{e-}05}}$ & 5003.82575$^{_{+0.00053}}_{^{-0.00053}}$ & $ $-0.114$^{_{+0.14}}_{^{-0.21}}$ & 2.13\\
 809.01 & $ $ 0.0168$^{_{+0.0064}}_{^{-0.0057}}$ & 0.177$^{_{+0.028}}_{^{-0.029}}$ & 0.1150$^{_{+0.0021}}_{^{-0.0026}}$ & 84.88$^{_{+1.5312}}_{^{-1.6161}}$ & $ $-0.022$^{_{+0.009}}_{^{-0.009}}$ & $ $-0.138$^{_{+0.152}}_{^{-0.168}}$ & 1.594732$^{_{+1.0\textrm{e-}05}}_{^{-1.0\textrm{e-}05}}$ & 5003.64776$^{_{+0.00034}}_{^{-0.00035}}$ & $ $ 0.046$^{_{+0.08}}_{^{-0.09}}$ & 1.62\\
 813.01 & $ $-0.0612$^{_{+0.0264}}_{^{-0.0308}}$ & 0.060$^{_{+0.014}}_{^{-0.012}}$ & 0.0813$^{_{+0.0063}}_{^{-0.0018}}$ & 89.17$^{_{+0.7397}}_{^{-0.9617}}$ & $ $-0.090$^{_{+0.011}}_{^{-0.008}}$ & $ $-0.381$^{_{+0.201}}_{^{-0.202}}$ & 3.895869$^{_{+9.5\textrm{e-}05}}_{^{-9.4\textrm{e-}05}}$ & 5003.52768$^{_{+0.00125}}_{^{-0.00123}}$ & $ $ 0.050$^{_{+0.04}}_{^{-0.03}}$ & 1.84\\
 830.01 & $ $-0.0000$^{_{+0.0000}}_{^{-0.0000}}$ & 0.095$^{_{+0.005}}_{^{-0.002}}$ & 0.1288$^{_{+0.0019}}_{^{-0.0019}}$ & 88.41$^{_{+0.7408}}_{^{-0.4725}}$ & $ $ 0.007$^{_{+0.314}}_{^{-0.340}}$ & $ $-0.106$^{_{+0.038}}_{^{-0.070}}$ & 3.525645$^{_{+1.2\textrm{e-}05}}_{^{-1.4\textrm{e-}05}}$ & 5003.04702$^{_{+0.00026}}_{^{-0.00028}}$ & $ $50.838$^{_{+101.}}_{^{-123.}}$ & 1.38\\
 838.01 & $ $ 0.0003$^{_{+0.0053}}_{^{-0.0025}}$ & 0.087$^{_{+0.004}}_{^{-0.005}}$ & 0.1254$^{_{+0.0555}}_{^{-0.0448}}$ & 86.89$^{_{+0.2559}}_{^{-0.2515}}$ & $ $-0.022$^{_{+0.053}}_{^{-0.054}}$ & $ $-0.449$^{_{+0.124}}_{^{-0.078}}$ & 4.859269$^{_{+1.2\textrm{e-}04}}_{^{-1.2\textrm{e-}04}}$ & 5006.01102$^{_{+0.00127}}_{^{-0.00129}}$ & $ $ 0.199$^{_{+5.84}}_{^{-1.14}}$ & 1.78\\
 840.01 & $ $ 0.0044$^{_{+0.0112}}_{^{-0.0101}}$ & 0.089$^{_{+0.008}}_{^{-0.009}}$ & 0.0959$^{_{+0.0030}}_{^{-0.0039}}$ & 87.10$^{_{+0.4951}}_{^{-0.4985}}$ & $ $ 0.084$^{_{+0.037}}_{^{-0.033}}$ & $ $-0.262$^{_{+0.172}}_{^{-0.119}}$ & 3.040329$^{_{+3.1\textrm{e-}05}}_{^{-3.0\textrm{e-}05}}$ & 5002.94877$^{_{+0.00053}}_{^{-0.00054}}$ & $ $-0.069$^{_{+0.45}}_{^{-1.31}}$ & 2.33\\
 843.01 & $ $ 0.0686$^{_{+0.0495}}_{^{-0.0253}}$ & 0.148$^{_{+0.025}}_{^{-0.014}}$ & 0.0507$^{_{+0.0037}}_{^{-0.0031}}$ & 83.64$^{_{+1.0836}}_{^{-1.7704}}$ & $ $ 0.060$^{_{+0.006}}_{^{-0.007}}$ & $ $ 0.065$^{_{+0.157}}_{^{-0.157}}$ & 4.190477$^{_{+1.7\textrm{e-}04}}_{^{-1.8\textrm{e-}04}}$ & 5004.44189$^{_{+0.00209}}_{^{-0.00221}}$ & $ $-0.056$^{_{+0.05}}_{^{-0.09}}$ & 1.74\\
 897.01 & $ $ 0.0132$^{_{+0.0076}}_{^{-0.0070}}$ & 0.122$^{_{+0.018}}_{^{-0.017}}$ & 0.1162$^{_{+0.0021}}_{^{-0.0025}}$ & 86.96$^{_{+0.7474}}_{^{-0.7906}}$ & $ $-0.001$^{_{+0.011}}_{^{-0.010}}$ & $ $-0.326$^{_{+0.155}}_{^{-0.136}}$ & 2.052357$^{_{+1.5\textrm{e-}05}}_{^{-1.4\textrm{e-}05}}$ & 5002.88992$^{_{+0.00036}}_{^{-0.00039}}$ & $ $-0.006$^{_{+0.16}}_{^{-0.18}}$ & 2.27\\
 908.01 & $ $ 0.0214$^{_{+0.0115}}_{^{-0.0191}}$ & 0.098$^{_{+0.028}}_{^{-0.017}}$ & 0.0794$^{_{+0.0031}}_{^{-0.0014}}$ & 88.24$^{_{+1.3406}}_{^{-2.2814}}$ & $ $-0.000$^{_{+0.016}}_{^{-0.014}}$ & $ $ 0.011$^{_{+0.198}}_{^{-0.223}}$ & 4.708359$^{_{+8.8\textrm{e-}05}}_{^{-8.2\textrm{e-}05}}$ & 5004.44499$^{_{+0.00092}}_{^{-0.00098}}$ & $ $-0.114$^{_{+0.09}}_{^{-0.18}}$ & 1.78\\
 913.01 & $ $ 0.0021$^{_{+0.0058}}_{^{-0.0025}}$ & 0.103$^{_{+0.013}}_{^{-0.010}}$ & 0.1112$^{_{+0.0016}}_{^{-0.0016}}$ & 88.15$^{_{+0.9795}}_{^{-0.7173}}$ & $ $-0.017$^{_{+0.040}}_{^{-0.043}}$ & $ $-0.069$^{_{+0.124}}_{^{-0.113}}$ & 4.082302$^{_{+3.1\textrm{e-}05}}_{^{-3.2\textrm{e-}05}}$ & 5002.63659$^{_{+0.00041}}_{^{-0.00040}}$ & $ $-0.604$^{_{+1.27}}_{^{-4.19}}$ & 1.82\\
1176.01 & $ $ 0.0002$^{_{+0.0061}}_{^{-0.0047}}$ & 0.119$^{_{+0.011}}_{^{-0.009}}$ & 0.1300$^{_{+0.0028}}_{^{-0.0033}}$ & 86.97$^{_{+0.5149}}_{^{-0.5087}}$ & $ $ 0.056$^{_{+0.121}}_{^{-0.126}}$ & $ $-0.234$^{_{+0.136}}_{^{-0.102}}$ & 1.973750$^{_{+9.6\textrm{e-}06}}_{^{-9.8\textrm{e-}06}}$ & 5011.68879$^{_{+0.00032}}_{^{-0.00030}}$ & $ $-0.088$^{_{+1.93}}_{^{-3.98}}$ & 1.87\\
1419.01 & $ $ 0.0225$^{_{+0.0451}}_{^{-0.0184}}$ & 0.256$^{_{+0.146}}_{^{-0.037}}$ & 0.0434$^{_{+0.0020}}_{^{-0.0015}}$ & 76.88$^{_{+7.5639}}_{^{-27.934}}$ & $ $-0.285$^{_{+0.826}}_{^{-0.468}}$ & $ $ 0.390$^{_{+0.214}}_{^{-0.299}}$ & 1.336130$^{_{+5.8\textrm{e-}05}}_{^{-5.5\textrm{e-}05}}$ & 5011.25785$^{_{+0.00480}}_{^{-0.00363}}$ & $ $-1.040$^{_{+0.97}}_{^{-1.81}}$ & 1.91\\
1459.01 & $ $ 0.0498$^{_{+0.0771}}_{^{-0.0366}}$ & 0.400$^{_{+0.021}}_{^{-0.025}}$ & 0.0917$^{_{+0.0365}}_{^{-0.0152}}$ & 69.02$^{_{+2.2429}}_{^{-0.8466}}$ & $ $ 0.004$^{_{+0.014}}_{^{-0.014}}$ & $ $ 0.027$^{_{+0.023}}_{^{-0.096}}$ & 0.692022$^{_{+9.3\textrm{e-}06}}_{^{-9.4\textrm{e-}06}}$ & 4966.11051$^{_{+0.00117}}_{^{-0.00119}}$ & $ $ 0.972$^{_{+1.13}}_{^{-0.58}}$ & 3.50\\
1541.01 & $ $ 0.0595$^{_{+0.0121}}_{^{-0.0113}}$ & 0.390$^{_{+0.004}}_{^{-0.005}}$ & 0.1693$^{_{+0.0008}}_{^{-0.0008}}$ & 82.85$^{_{+0.6984}}_{^{-0.6713}}$ & $ $-0.012$^{_{+0.012}}_{^{-0.008}}$ & $ $ 0.775$^{_{+0.018}}_{^{-0.017}}$ & 2.379293$^{_{+1.5\textrm{e-}05}}_{^{-1.5\textrm{e-}05}}$ & 4966.65019$^{_{+0.00052}}_{^{-0.00058}}$ & $ $ 0.485$^{_{+0.13}}_{^{-0.14}}$ & 28.7\\
1543.01 & $ $ 0.1226$^{_{+0.0745}}_{^{-0.0324}}$ & 0.369$^{_{+0.008}}_{^{-0.004}}$ & 0.1426$^{_{+0.0021}}_{^{-0.0014}}$ & 86.08$^{_{+3.0058}}_{^{-22.298}}$ & $ $ 0.114$^{_{+0.010}}_{^{-0.011}}$ & $ $ 0.797$^{_{+0.010}}_{^{-0.007}}$ & 3.964230$^{_{+8.1\textrm{e-}05}}_{^{-7.4\textrm{e-}05}}$ & 4969.03213$^{_{+0.00187}}_{^{-0.00315}}$ & $ $-0.065$^{_{+0.19}}_{^{-0.20}}$ & 38.4\\

  \enddata
  \label{modelresultstab}
\end{deluxetable}


